\section{XcalableMP element}

\subsection{ {\tt coArrayType} element}

Represents a co-array type declared by "{\tt \#pragma xmp coarray}". This element has the following attributes:

\begin{itemize}
\item {\tt type} - derived data type name,
\item {\tt element\_type} - co-array element data type name, and
represents a co-array type of two or more dimensions when the type corresponding to the data type name is {\tt coArrayType}.
\item {\tt array\_size} - represents the co-array dimension.
\end{itemize}

This element has the following subelement:

\begin{itemize}
\item {\tt arraySize} - represents the co-array dimension.
When there is an {\tt arraySize} element, the {\tt array\_size} attribute value should be "{\tt *}". 
\end{itemize}

\subsubsection*{Example}
\vspace{1mm}

\begin{CExample}
int A[10];
#pragma xmp coarray [*][2]::A
\end{CExample}

The element that represents the type for variable {\tt A} above is {\tt coArrayType} {\tt C2} given below.
\vspace{2mm}

\begin{XcodeMLExample}
<arrayType type="A1" element_type="int" array_size="10"/>
<coArrayType type="C1" element_type="A1"/>
<coArrayType type="C2" element_type="C1" array_size="2"/>
\end{XcodeMLExample}


\subsection{ {\tt coArrayRef} element}

The element represents a reference to a co-array type variable.
This element has the following subelements:

\begin{itemize}
\item First expression - represents the co-array variable expression and
\item Second expression - represents the expression for the co-array dimension. 
Specify more than one expression if there are multiple dimensions. 
\end{itemize}


\subsection{ {\tt subArrayRef} element}

Represents a reference to a subarray. 
This element has the following subelements. Subelements cannot be omitted.

\begin{itemize}
\item The first element has the expression that represents the array.
\item {\tt lowerBound} - represents the lower limit of the index. 
This subelement has an expression element.
\item {\tt upperBound} - represents the upper limit of the index.
This subelement has an expression element.
\item {\tt step} - represents the step size for the index. 
This subelement has an expression element.
\end{itemize}


