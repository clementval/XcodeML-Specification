\section{Symbol list}

\subsection{ {\tt id} element}

The {\tt id} element defines the variable names, array names, function names, {\tt struct}/{\tt union} member names, function arguments, and compound statement local variable names. The {\tt id} element has the following attributes:

\begin{itemize}
\item {\tt sclass} - represents one of the storage classes: '{\tt auto}', '{\tt param}', '{\tt extern}', '{\tt extern\_def}', '{\tt static}', '{\tt register}', '{\tt label}', '{\tt tagname}', '{\tt moe}', or '{\tt typedef\_name}';
\item {\tt type} - represents the identifier data type;
\item {\tt bit\_field} attribute - specifies the member bit field in the {\tt structType} and {\tt unionType} elements;
\item {\tt is\_gccThread} - specifies whether or not the GCC {\tt \_thread} keyword is defined using {\tt 0} or {\tt 1} ({\tt false} or {\tt true}); and
\item {\tt is\_gccExtension} attribute.
\end{itemize}

The element has the following elements:

\begin{itemize}
\item {\tt name} element - specifies the names of identifiers,
\item {\tt value} element - specifies the value corresponding to the identifier, and
\item {\tt bitField} element - specifies the member bit field in the {\tt structType} and {\tt unionType} elements.
\end{itemize}

If the identifier is a variable, it has an element for the address. However, there is no need for the address element if the variable is created by the compiler.

\subsubsection*{Example}

The symbol table entry for the variable "{\tt xyz}" in "{\tt int xyz}" is as follows. It can be noted that {\tt P6e7e0} is the {\tt type\_id} for "{\tt int *}".
\vspace{2mm}

\begin{XcodeMLExample}
  <id sclass="extern_def" type="int"> 
   <name>xyz</name>
   <value>
     <VarAddr type="P6e7e0">xyz</varAddr>
   </value>
  </id>
\end{XcodeMLExample}

The symbol table entry for the "{\tt foo}" function in "{\tt int foo()}" is as follows. It can be noted that {\tt F6f168} is the {\tt type\_id} corresponding to the "{\tt foo}" data type, and {\tt P6f1a8} is the {\tt type\_id} of the pointer to {\tt F6f168}. Furthermore, the {\tt foo} identifier becomes the pointer to the function.
\vspace{2mm}

\begin{XcodeMLExample}
  <id sclass="extern_def" type="0x6f168">
   <name>foo</name>
   <value>
    <funcAddr type="0xfla8">foo</funcAddr>
   </value>
  </id>
\end{XcodeMLExample}


\subsection{ {\tt globalSymbols} element}

Defines identifiers that have global scope. The element has {\tt id} elements for identifiers with global scope.

\subsection{ {\tt symbols} element}

The symbols element defines identifiers that have local scope. The element has {\tt id} elements that correspond to definition identifiers. 


