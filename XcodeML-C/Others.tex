\section{Other elements and attributes}

\subsection{ {\tt typeName} element}

The {\tt typeName} element represents the name of the type. Various subelements are specified, such as {\tt sizeOfExpr}, {\tt gccAlignOfExpr}, and {\tt builtin\_op}. The attribute contains the type that indicates the data type name identifier.

\subsection{ {\tt is\_gccExtension} attribute}

The {\tt is\_gccExtension} attribute defines whether or not the GCC {\tt \_extension\_} keyword is added to the beginning of the element. The attribute has a value of {\tt 0} or {\tt 1} ({\tt false} or {\tt true}). The {\tt is\_gccExtension} attribute can be omitted, in which case it is the same as assigning a 0 value. The following elements can have the {\tt is\_gccExtension} attribute:

\begin{itemize}
\item {\tt id}
\item {\tt functionDefinition}
\item {\tt castExpr}
\item {\tt gccAsmDefinition}
\end{itemize}

\subsubsection*{Example}

The definition for "{\tt \_\_extension\_\_ typedef long long int64\_t}" corresponds to the following:
\vspace{2mm}

\begin{XcodeMLExample}
 <id type="long_long" sclass="typedef" is_gccExtension="1">
   <name>int64_t</name>
 </id>
\end{XcodeMLExample}

\subsection{ {\tt gccAsm}, {\tt gccAsmDefinition}, and {\tt gccAsmStatement} elements}

The {\tt gccAsm, {\tt gccAsmDefinition}, and {\tt gccAsmStatement} elements define the GCC {\tt asm} and {\tt \_\_asm\_\_} keywords. The elements have the asm variable string as an element.

\begin{itemize}
\item {\tt gccAsm} - represents the {\tt asm} expression. This element has the subelement given below:
  \begin{itemize}
  \item {\tt stringConstant} (1 item) - represents assembly code,
  \end{itemize}
\item {\tt gccAsmDefinition} - represents the {\tt asm} definition, 
  The subelements are the same as for {\tt gccAsm}.
\item {\tt gccAsmStatement} - represents the {\tt asm} statement.
  The element has the following attribute:
  \begin{itemize}
  \item {\tt is\_volatile} - indicates whether or not {\tt volatile} is specified; uses {\tt 0} or {\tt 1} for {\tt false} or {\tt true}, respectively.
  \end{itemize}
  The element has the following subelements:
  \begin{itemize}
  \item {\tt stringConstant} (1 item) - represents assembly code.
  \item {\tt gccAsmOperands} (2 items) - The first item represents the output operand, while the second represents the input operand. If the operands are omitted, a tag that does not have a subelement is described. The subelement has {\tt gccAsmOperand} (multiple items) as a subelement.
  \item {\tt gccAsmClobbers} (0 -1 item) - represents clobber values. 
  The subelement has zero or more {\tt stringConstant} elements as subelements.
  \item {\tt gccAsmOperand} - represents the input and output operands. 
    The element has the following attributes: 
    \begin{itemize}
    \item {\tt match} (can be omitted)  - represents the identifier specified instead of the matching constraint (corresponds to "{\tt [identifier]}") and
    \item {\tt constraint} (can be omitted)  - represents the {\tt constraint/constraint} modifier.
    \end{itemize}
    The element has the following subelement:
    \begin{itemize}
    \item Expression (1 item) - represents the expression that defines whether the element is input or output.
    \end{itemize}
  \end{itemize}
\end{itemize}

\subsubsection*{Example}
\vspace{2mm}

\begin{CExample}
  asm volatile (
       "661:\n"
       "\tmovl %0, %1\n662:\n"
       ".section .altinstructions,\"a\"\n"
       ".byte %c[feat]\n"
       ".previous\n"
       ".section .altinstr_replacement,\"ax\"\n"
       "663:\n"
       "\txchgl %0, %1\n"
       : "=r" (v), "=m" (*addr)
       : [feat] "i" (115), "0" (v), "m" (*addr));
\end{CExample}
\vspace{1mm}

\begin{XcodeMLExample}
   <gccAsmStatement is_volatile="1">
     <stringConstant><![CDATA[661:\n\tmovl ...]]></stringConstant>
     <gccAsmOperands>
       <gccAsmOperand constraint="=r">
         <Var>v</Var>
       </gccAsmOperand>
       <gccAsmOperand constraint="=m">
         <pointerRef><Var>addr</Var></pointerRef>
       </gccAsmOperand>
     </gccAsmOperands>
     <gccAsmOperands>
       <gccAsmOperand match="feat" constraint="i">
         <intConstant>115</intConstant>
       </gccAsmOperand>
       <gccAsmOperand constraint="m">
         <pointerRef><Var>addr</Var></pointerRef>
       </gccAsmOperand>
     </gccAsmOperands>
   </gccAsmStatement>
\end{XcodeMLExample}


\subsection{ {\tt gccAttributes} element}

The {\tt gccAttributes} element defines the GCC {\tt \_\_attribute\_\_} keyword. The element has the {\tt \_\_attribute\_\_} argument character string. The {\tt gccAttributes} element has multiple {\tt gccAttribute} elements as subelements.

\begin{itemize}
\item Elements that represent a type all have the {\tt gccAttributes} element as a subelement (0 - 1 item).
\item The id element has the {\tt gccAttributes} element as a subelement (0 - 1 item).
\item The {\tt functionDefinition} element has the {\tt gccAttributes} element as a subelement (0 - 1 item).
\end{itemize}

\subsubsection*{Example}

This example sets the {\tt gccAttributes} subelements for an element that represents a type.
\vspace{2mm}

\begin{CExample}
  typedef __attribute__((aligned(8))) int ia8_t;
  ia8_t __attribute__((aligned(16)) n;
\end{CExample}
\vspace{1mm}

\begin{XcodeMLExample}
 <typeTable>
    <basicType type="B0" name="int" align="8" size="4"/>
      <gccAttributes>
        <attribute>aligned(8)</attribute>
     </gccAttributes>
    </basicType>
    <basicType type="B1" name="int" align="16" size="4"/>
     <gccAttributes>
        <attribute>aligned(8)</attribute>
        <attribute>aligned(16)</attribute>
      </gccAttributes>
   </basicType>
  </typeTable>
 <globalSymbols>
   <id type="B0" sclass="typedef_name">
      <name>ia8_t</name>
    </id>
   <id type="B1">
      <name>n</name>
    </id>
  </globalSymbols>
 <globalDeclarations>
    <varDecl>
      <name>n</name>
    </varDecl>
  </globalDeclarations>
\end{XcodeMLExample}

This example sets the {\tt gccAttributes} of the subelements (0 - 1 item) for the id element and the {\tt functionDefinition} element.
\vspace{2mm} 

\begin{CExample}
 void func(void);
 void func2(void) __attribute__(alias("func")); 
  
  void __attribute__((noreturn)) func() {
     ...
  }
\end{CExample}
\vspace{1mm} 
 
\begin{XcodeMLExample}
 <typeTable>
   <functionType type="F0">
     <params>
        <name type="void"/>
      </params>
    </functionType>
    <functionType type="F1">
      <params>
        <name type="void"/>
      </params>
    </functionType>
  </typeTable>
  <globalSymbols>
    <id type="F0" sclass="extern_def">
      <name>func</name>
    </id>
    <id type="F1" sclass="extern_def">
      <name>func2</name>
      <gccgccAttributes>
        <gccAttribute>alias("func")</gccAttribute>
      </gccgccAttributes>
    </id>
  </globalSymbols>
  <globalDeclarations>
    <functionDefinition>
      <name>func</name>
      <gccgccAttributes>
        <gccAttribute>noreturn</gccAttribute>
      </gccgccAttributes>
      <body>...</body>
    </functionDefinition>
  </globalDeclarations>
\end{XcodeMLExample}


\subsection{ {\tt builtin\_op} element}

The {\tt builtin\_op} element represents a call to an intrinsic compiler. The element has the following elements, each having values from 0 to multiple items. The order of the subelements must match the order of the function arguments.

\begin{itemize}
\item {\tt expression} - specifies the expression as an argument for the function that is called;
\item {\tt typeName} - specifies the type name as an argument for the function that is called; and
\item {\tt gccMemberDesignator} - specifies the structure or union member designator as an argument for the function that is called.
The element has two attributes: ref, which indicates the structure or union derived data type name; and member, which indicates the member designator character string. The element has an expression for the array index (0 -1 item) and the {\tt gccMemberDesignator} element (0 -1 item) as subelements.
\end{itemize}


\subsection{ {\tt is\_gccSyntax} attribute}

The {\tt is\_gccSyntax} attribute defines whether or not the expression, statement, or declaration corresponding to a tag uses the GCC extension. The value is {\tt 0} or {\tt 1} ({\tt false} or {\tt true}). This attribute can be omitted, in which case it is the same as assigning a 0 value.


\subsection{ {\tt is\_modified} attribute}

The {\tt is\_modified} attribute defines whether or not the expression, statement, or declaration corresponding to a tag is modified during compilation. The value is {\tt 0} or {\tt 1} ({\tt false} or {\tt true}). This attribute can be omitted, in which case it is the same as assigning a 0 value.
The following elements can have the {\tt is\_gccSyntax} or {\tt is\_modified} attribute:

\begin{itemize}
\item {\tt varDecl},
\item Statement elements, and
\item Expression elements.
\end{itemize}


