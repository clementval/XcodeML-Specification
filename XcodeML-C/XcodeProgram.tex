\section{ {\tt XcodeProgram} element}

Programs written in Xcode are constructed from external definitions, comprising a Type table and a global Id table. The top-level element in an Xcode file is the XcodeProgram element. The XcodeProgram element includes the following elements:

\begin{itemize}
\item typeTable element - information on data type used by the program,
\item globalSymbols element - information on global variables used by the program, and
\item globalDeclarations element - information about function and variable declarations.
\end{itemize}

The elements have the following information attributes:

\begin{itemize}
\item {\tt compiler-info} - C-to-C compiler information,
\item {\tt version} - C-to-C compiler version information,
\item {\tt time} - Date and time of compilation,
\item {\tt language} - source language information, and
\item {\tt source} - source information.
\end{itemize}

\subsection{ {\tt name} element}

This element is used to specify a name, such as a variable name or a type name. The name is a character string. The element attributes are type attributes. The attribute values are type identifiers.

\subsection{ {\tt value} element}

This element is used to specify initial values. This element has the following subelements.

\begin{itemize}
\item {\tt expression} (0 - multiple items) - expression that specifies a value
\item {\tt value} - nested value. This corresponds to "{…}".
\end{itemize}

\subsubsection*{Example}

The following expression sets the initial value of an int type to 1.
\vspace{2mm}

\begin{CExample}
<value>
  <intConstant type="int">1</intConstant>
</value>
\end{CExample}

The following expression sets the initial value of an int type vector to {1,2}.
\vspace{2mm}

\begin{XcodeMLExample}
<value>
  <value>
    <intConstant type="int">1</intConstant>
    <intConstant type="int">2</intConstant>
  </value>
</value>
\end{XcodeMLExample}


