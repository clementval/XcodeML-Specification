\section{Definition and declaration element}

\subsection{ {\tt globalDeclarations} element}

The {\tt globalDeclarations} element is the element to declare external variables and to define functions in the program.

%\subsubsection*{Contents model}
%{\tt
%(FfunctionDefinition | FmoduleDefinition)*
%}
\XcodeMLContentsModel{ (FfunctionDefinition | FmoduleDefinition)* }

\begin{XcodeMLChildElements}
\XcodeMLElementDef{FfunctionDefinition}
{the definition of a Fortran 90 main program, function or subroutine.}{O}
\XcodeMLElementDef{FmoduleDefinition}
{the definition of a Fortran 90 module.}{O}
\end{XcodeMLChildElements}

\begin{XcodeMLAttributes}
\XcodeMLAttrDef{-}{-}
{-}{-}
\end{XcodeMLAttributes}


\subsection{ {\tt declarations} element}

The {\tt declarations} element is the element to declare variables etc. in the program.

%\subsubsection*{Contents model}
%{\tt
%(varDecl | FstructDecl | externDecl | FuseDecl | FuseOnlyDecl | FinterfaceDecl | FnamelistDecl | FequivalenceDecl | FcommonDecl)*
%}
\XcodeMLContentsModel{ (varDecl | FstructDecl | externDecl | FuseDecl | FuseOnlyDecl | FinterfaceDecl | FnamelistDecl | FequivalenceDecl | FcommonDecl | FimportDecl)* }

\begin{XcodeMLChildElements}
\XcodeMLElementDef{varDecl}
{the definition of a variable}{O}
\XcodeMLElementDef{FstructDecl}
{the Fortran 90 derived type definition}{O}
\XcodeMLElementDef{externDecl}
{the definition of an external variable or an identifier}{O}
\XcodeMLElementDef{FuseDecl}
{the Fortran 90 {\tt use} declaration}{O}
\XcodeMLElementDef{FuseOnlyDecl}
{the Fortran 90 {\tt use} declaration with the {\tt only} option}{O}
\XcodeMLElementDef{FinterfaceDecl}
{the Fortran 90 {\tt interface} statement}{O}
\XcodeMLElementDef{FnamelistDecl}
{the Fortran 90 {\tt namelist} statement}{O}
\XcodeMLElementDef{FequivalenceDecl}
{the Fortran 90 {\tt equivalence} statement}{O}
\XcodeMLElementDef{FcommonDecl}
{the Fortran 90 {\tt common} statement}{O}
\XcodeMLElementDef{FimportDecl}
{the Fortran 2003 {\tt import}  statement}{O}
\end{XcodeMLChildElements}

\begin{XcodeMLAttributes}
\XcodeMLAttrDef{-}{-}
{-}{-}
\end{XcodeMLAttributes}


\subsection{ {\tt FfunctionDefinition} element}

The {\tt FfunctionDefinition} element defines a program, function or subroutine.
Differently from C language, arguments in Fortran are passed by references, so we prepare another entry than functionDefinition of XcodeML.

%\subsubsection*{Contents model}
%{\tt
%(name, symbols?, params?, declarations?, body)
%}
\XcodeMLContentsModel{ (name, symbols?, params?, declarations?, body) }

\begin{XcodeMLChildElements}
\XcodeMLElementDef{name}
{Specifies the name of  the function or  the subroutine.}{R}
\XcodeMLElementDef{symbols}
{Specifies the symbols included in the element.}{O}
\XcodeMLElementDef{params}
{Specifies the dummy arguments.}{O}
\XcodeMLElementDef{declarations}
{Specifies the definitions and declarations included in the element.}{O}
\XcodeMLElementDef{body}
{Specifies the executable statemtents included in the element.}{R}
\end{XcodeMLChildElements}

\begin{XcodeMLAttributes}
\XcodeMLAttrDef{common attributes}{-}
{Refer to "\ref{sec:Commonattributesofdefinition} Common attributes of the definition / declaration / statement element".}{-}
\end{XcodeMLAttributes}


\subsection{ {\tt varDecl} element}

The {\tt varDecl} element declares a variable.

%\subsubsection*{Contents model}
%{\tt
%(name, value?)
%}
\XcodeMLContentsModel{ (name, value?) }

\begin{XcodeMLChildElements}
\XcodeMLElementDef{name}
{Specifies the name of the variable.}{R}
\XcodeMLElementDef{value}
{Specifies the initial value if present.}{O}
\end{XcodeMLChildElements}

\begin{XcodeMLAttributes}
\XcodeMLAttrDef{common attributes}{-}
{Refer to "\ref{sec:Commonattributesofdefinition} Common attributes of the definition / declaration / statement element".}{-}
\end{XcodeMLAttributes}


\subsection{ {\tt FstructDecl} element}

The {\tt FstructDecl} element defines the derived type.

%\subsubsection*{Contents model}
%{\tt
%(name)
%}
\XcodeMLContentsModel{ (name) }

\begin{XcodeMLChildElements}
\XcodeMLElementDef{name}
{Specifies the name of the derived type.}{R}
\end{XcodeMLChildElements}

\begin{XcodeMLAttributes}
\XcodeMLAttrDef{common attributes}{-}
{Refer to "\ref{sec:Commonattributesofdefinition} Common attributes of the definition / declaration / statement element".}{-}
\end{XcodeMLAttributes}


\subsection{ {\tt externDecl} element}

The {\tt externDecl} element declares an external definition.

%\subsubsection*{Contents model}
%{\tt
%(name)
%}
\XcodeMLContentsModel{ (name) }

\begin{XcodeMLChildElements}
\XcodeMLElementDef{name	}
{Specifies the name of the identifier of the external definition to declare.}{R}
\end{XcodeMLChildElements}

\begin{XcodeMLAttributes}
\XcodeMLAttrDef{common attributes}{-}
{Refer to "\ref{sec:Commonattributesofdefinition} Common attributes of the definition / declaration / statement element".}{-}
\end{XcodeMLAttributes}


\subsection{ {\tt FmoduleDefinition} element}

The {\tt FmoduleDefinition} element represents a {\tt module} statement.

%\subsubsection*{Contents model}
%{\tt
%(symbols?, declarations?, FcontainsStatement?)
%}
\XcodeMLContentsModel{ (symbols?, declarations?, FcontainsStatement?) }

\begin{XcodeMLChildElements}
\XcodeMLElementDef{symbols}
{Specifies the symbols included in the element.}{O}
\XcodeMLElementDef{declarations}
{Specifies the definitions and declarations included in the element.}{O}
\XcodeMLElementDef{FcontainsStatement}
{Specifies the {\tt contain} statement.}{O}
\end{XcodeMLChildElements}

\begin{XcodeMLAttributes}
\XcodeMLAttrDef{common attributes}{-}
{Refer to "\ref{sec:Commonattributesofdefinition} Common attributes of the definition / declaration / statement element".}{-}
\XcodeMLAttrDef{name}{text}
{Specified the module name.}{R}
\XcodeMLAttrDef{is\_sub}{bool}
{{\tt true} if the element represents a submodule.}{O}
\XcodeMLAttrDef{parent\_name}{text}
{Specifies the name of the parent module.}{O}
\end{XcodeMLAttributes}


\subsection{ {\tt FuseDecl} element}

The {\tt FuseDecl} element represents an {\tt use} statement without the {\tt only} option.

%\subsubsection*{Contents model}
%{\tt
%(rename*)
%}
\XcodeMLContentsModel{ (rename*) }

\begin{XcodeMLChildElements}
\XcodeMLElementDef{rename}
{Specifies the rename.}{O}
\end{XcodeMLChildElements}

\begin{XcodeMLAttributes}
\XcodeMLAttrDef{common attributes}{-}
{Refer to "\ref{sec:Commonattributesofdefinition} Common attributes of the definition / declaration / statement element".}{-}
\XcodeMLAttrDef{name}{text}
{Specifies the module name.}{R}
\XcodeMLAttrDef{intrinsic}{bool}
{{\tt true} if the {\tt intrinsic} attribute is specified.
 {\tt false} if the {\tt non\_intrinsic} attribute is specified.}{O}
\end{XcodeMLAttributes}


\subsection{ {\tt FuseOnlyDecl} element}

The {\tt FuseOnlyDecl} element represents an {\tt use} statement with the {\tt only} option.

%\subsubsection*{Contents model}
%{\tt
%(renamable*)
%}
\XcodeMLContentsModel{ (renamable*) }

\begin{XcodeMLChildElements}
\XcodeMLElementDef{renamable}
{Specifies the {\tt only} list.}{O}
\end{XcodeMLChildElements}

\begin{XcodeMLAttributes}
\XcodeMLAttrDef{common attributes}{-}
{Refer to "\ref{sec:Commonattributesofdefinition} Common attributes of the definition / declaration / statement element".}{-}
\XcodeMLAttrDef{name}{text}
{Specifies the module name.}{R}
\XcodeMLAttrDef{intrinsic}{bool}
{{\tt true} if the {\tt intrinsic} attribute is specified.
 {\tt false} if the {\tt non\_intrinsic} attribute is specified.}{O}
\end{XcodeMLAttributes}


\subsection{ {\tt FinterfaceDecl} element}

The {\tt FinterfaceDecl} element represents an {\tt interface} statement.

%\subsubsection*{Contents model}
%{\tt
%(FmoduleProcedureDecl | FfunctionDecl)*
%}
\XcodeMLContentsModel{ (FmoduleProcedureDecl | FfunctionDecl)* }

\begin{XcodeMLChildElements}
\XcodeMLElementDef{FmoduleProcedureDecl}
{Specifies the {\tt module procedure} statements included in the element.}{O}
\XcodeMLElementDef{FfunctionDecl}
{Specifies the functions and subroutines included in the element.}{O}
\end{XcodeMLChildElements}

\begin{XcodeMLAttributes}
\XcodeMLAttrDef{common attributes}{-}
{Refer to "\ref{sec:Commonattributesofdefinition} Common attributes of the definition / declaration / statement element".}{-}
\XcodeMLAttrDef{name}{text}
{Specifies the generic name or the defined operator name.}{O}
\XcodeMLAttrDef{is\_operator}{bool}
{Specifies if the interface is a defined operator.}{O}
\XcodeMLAttrDef{is\_assignment}{bool}
{Specifies if the interface is a defined assignment operator.}{O}
\XcodeMLAttrDef{is\_defined\_io}{bool}
{Either of "{\tt read(formatted)}", "{\tt read(unformatted)}",
 "{\tt write(formatted)}" or "{\tt write(unformatted)}" is specified if the element represents
 a user-defined I/O procedure.}{O}
\XcodeMLAttrDef{is\_abstract}{bool}
{{\tt true} if the element represents an abstract interface.}{O}
\end{XcodeMLAttributes}


\subsection{ {\tt FmoduleProcedureDecl} element}

The {\tt FmoduleProcedureDecl} represents a {\tt module procedure} statement in an {\tt interface} statement.

%\subsubsection*{Contents model}
%{\tt
%(name*)
%}
\XcodeMLContentsModel{ (name*) }

\begin{XcodeMLChildElements}
\XcodeMLElementDef{name}
{Specifies the procesure names.}{O}
\end{XcodeMLChildElements}

\begin{XcodeMLAttributes}
\XcodeMLAttrDef{common attributes}{-}
{Refer to "\ref{sec:Commonattributesofdefinition} Common attributes of the definition / declaration / statement element".}{-}
\end{XcodeMLAttributes}


\subsection{ {\tt FfunctionDecl} element}

The {\tt FfunctionDecl} element represents a {\tt function} or {\tt subroutine} statement in an {\tt interface} statement.

%\subsubsection*{Contents model}
%{\tt
%(name, params?, declarations?)
%}
\XcodeMLContentsModel{ (name, symbols?, declarations?) }

\begin{XcodeMLChildElements}
\XcodeMLElementDef{name}
{Specifies the name of the function or the subroutine.}{R}
\XcodeMLElementDef{symbols}
{Specifies the symbols included in the element.}{O}
\XcodeMLElementDef{declarations}
{Specifies the definitions and declarations in the element.}{O}
\end{XcodeMLChildElements}

\begin{XcodeMLAttributes}
\XcodeMLAttrDef{common attributes}{-}
{Refer to "\ref{sec:Commonattributesofdefinition} Common attributes of the definition / declaration / statement element".}{-}
\end{XcodeMLAttributes}


\subsection{ {\tt FimportDecl} element}

The {\tt FimportDecl} element represents an {\tt import} statement.

%\subsubsection*{Contents model}
%{\tt
%(name*)
%}
\XcodeMLContentsModel{ (name*) }

\begin{XcodeMLChildElements}
\XcodeMLElementDef{name}
{Specifies the name to import.}{O}
\end{XcodeMLChildElements}

\begin{XcodeMLAttributes}
\XcodeMLAttrDef{common attributes}{-}
{Refer to "\ref{sec:Commonattributesofdefinition} Common attributes of the definition / declaration / statement element".}{-}
\end{XcodeMLAttributes}


\subsection{ {\tt FenumDecl} element}

The {\tt FenumDecl} element represents an {\tt enum} statement.

\XcodeMLContentsModel{ empty }

\begin{XcodeMLChildElements}
  \XcodeMLElementDef{-}
  {-}{-}
\end{XcodeMLChildElements}

\begin{XcodeMLAttributes}
\XcodeMLAttrDef{type}{text}
{Specifies the type name identifier in XcodeML/Fortran.}{R}
\XcodeMLAttrDef{common attributes}{-}
{Refer to "\ref{sec:Commonattributesofdefinition} Common attributes of the definition / declaration / statement element".}{-}
\end{XcodeMLAttributes}

\subsection{ {\tt FmoduleProcedureDefinition} element}

The {\tt FmoduleProcedureDefinition} element represents a separate module procedure.

\XcodeMLContentsModel{ (name, symbols?, params?, declarations?, body) }

\begin{XcodeMLChildElements}
\XcodeMLElementDef{name}
{Specifies the name of  the separate module procedure.}{R}
\XcodeMLElementDef{symbols}
{Specifies the symbols included in the element.}{O}
\XcodeMLElementDef{params}
{Specifies the dummy arguments.}{O}
\XcodeMLElementDef{declarations}
{Specifies the definitions and declarations included in the element.}{O}
\XcodeMLElementDef{body}
{Specifies the executable statemtents included in the element.}{R}
\end{XcodeMLChildElements}

\begin{XcodeMLAttributes}
\XcodeMLAttrDef{common attributes}{-}
{Refer to "\ref{sec:Commonattributesofdefinition} Common attributes of the definition / declaration / statement element".}{-}
\end{XcodeMLAttributes}

