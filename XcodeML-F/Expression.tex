\section{Expression element}

This chapter describes the XML elements corresponding to the syntax elements for expressions in Fortran.
Each element has a {\tt type} element attached, which can be used to extract the data type information of the expression.

\subsection{Constant element}

The elements representing constants are described below.

\subsubsection{ {\tt FintConstant, FrealConstant, FcharacterConstant, FlogicalConstant } element}

The {\tt FintConstant}, {\tt FrealConstant}, {\tt FcharacterConstant} and {\tt FlogicalConstant} element represents a constant of the {\tt integer}, {\tt real}, {\tt character} and {\tt logical} type respectively.
\newline

The contents of the elements are shown below.
\newline

\begin{XcodeMLElementList}
\XcodeMLElementFormat{FintConstant}
{Specifies a constant that has an integer value;
 a decimal or hexadecimal (starting with "{\tt 0x}") number.}
\XcodeMLElementFormat{FrealConstant}
{32bit hexadeciaml(starting with 0x) two numbers representing an IEEE754 floating-point number.}
\XcodeMLElementFormat{FcharacterConstant}
{A character string.}
\XcodeMLElementFormat{FlogicalConstant}
{{\tt 1}(or {\tt true}) representing true of the logical value, or {\tt 0}(or {\tt false}) representing false.}
\end{XcodeMLElementList}

%\subsubsection*{Contents model}
%{\tt
%(\#PCDATA)
%}
\XcodeMLContentsModel{ (\#PCDATA) }

\begin{XcodeMLChildElements}
\XcodeMLElementDef{-}
{-}{-}
\end{XcodeMLChildElements}

\begin{XcodeMLAttributes}
\XcodeMLAttrDef{type}{text}
{Specifies the type name identifier in XcodeML/Fortran.
 Refer to "\ref{sec:Thedatatypenameidentifier} The data type name identifier".}{O}
\XcodeMLAttrDef{kind}{text}
{Specifies the kind parameter of the type element.}{O}
\end{XcodeMLAttributes}


\subsubsection{ {\tt FcomplexContant} element}

The {\tt FcomplexContant} element represents a constant of the {\tt complex} type.

%\subsubsection*{Contents model}
%{\tt
%(FrealConstant, FrealConstant)
%}
\XcodeMLContentsModel{ (FrealConstant, FrealConstant) }

\begin{XcodeMLChildElements}
\XcodeMLElementDef{FrealConstant}
{Specifies a couple of a constant of the {\tt Freal} type.}{R}
\end{XcodeMLChildElements}

\begin{XcodeMLAttributes}
\XcodeMLAttrDef{type}{text}
{Specifies the type name identifier in XcodeML/Fortran.
 Refer to "\ref{sec:Thedatatypenameidentifier} The data type name identifier".}{O}
\end{XcodeMLAttributes}


\subsection{Array constructor element}

\subsubsection{ {\tt FarrayConstructor} element}

The {\tt FarrayConstructor} element represents an array constructor.

%\subsubsection*{Contents model}
%{\tt
%(exprModel)*
%}
\XcodeMLContentsModel{ (exprModel)* }

\begin{XcodeMLChildElements}
\XcodeMLElementDef{exprModel}
{Specifies the expression of the value of the array element. Refer to "\ref{sec:exprModel} exprModel".}{O}
\end{XcodeMLChildElements}

\begin{XcodeMLAttributes}
\XcodeMLAttrDef{type}{text}
{Specifies the type name identifier in XcodeML/Fortran.
 Refer to "\ref{sec:Thedatatypenameidentifier} The data type name identifier".}{O}
\end{XcodeMLAttributes}


\subsection{Structure constructor element}

\subsubsection{ {\tt Constant} element}

The {\tt FstructConstructor} element represents a structure constructor.

%\subsubsection*{Contents model}
%{\tt
%(typeParamValues?, (exprModel)*)
%}
\XcodeMLContentsModel{ (typeParamValues?, (exprModel)*) }

\begin{XcodeMLChildElements}
\XcodeMLElementDef{typeParamValues}
{Specifies the value of the type parameter.}{O}
\XcodeMLElementDef{exprModel}
{Specifies the expressions representing the values of the structure components. Refer to "\ref{sec:exprModel} exprModel".}{R}
\end{XcodeMLChildElements}

\begin{XcodeMLAttributes}
\XcodeMLAttrDef{type}{text}
{Specifies the type name identifier in XcodeML/Fortran.
 Refer to "\ref{sec:Thedatatypenameidentifier} The data type name identifier".}{O}
\end{XcodeMLAttributes}

\subsubsection*{Example}

The {\tt FstructConstructor} element for the type {\tt S} below is as follows:
\vspace{2mm}

\begin{Fexample2008}
type S(alen,blen,clen)
real,length:: alen,blen,clen
real::a(alen) = 0.0
real::b(blen) = 0.0
real::b(clen) = 0.0
end type
... = S(100,100,100)(b=1.0,c=2.0)
\end{Fexample2008}
\vspace{1mm}

\begin{XcodeMLFExample}
<FstructConstructor type="TYPE_ID_OF_S">
<typeParamValues>
  <FintConstant type="Fint">100</FintConstant>
<FintConstant type="Fint">100</FintConstant>
<FintConstant type="Fint">100</FintConstant>
  </typeParamValues>
  <namedValue name="b">
     <FrealConstant>1.0</FrealConstant>
  </namedValue>
  <namedValue name="c">
     <FrealConstant>2.0</FrealConstant>
  </namedValue>
</FstructConstructor>
\end{XcodeMLFExample}


\subsection{Elements referencing variables}

The elements representing a reference to a variable are described below.

\subsubsection{ {\tt Var} element}

The {\tt Var} element represents a name of the variable.
Specifies the name of the variable as the content.

%\subsubsection*{Contents model}
%{\tt
%(\#PCDATA)
%}
\XcodeMLContentsModel{ (\#PCDATA) }

\begin{XcodeMLChildElements}
\XcodeMLElementDef{-}
{-}{-}
\end{XcodeMLChildElements}

\begin{XcodeMLAttributes}
\XcodeMLAttrDef{type}{text}
{Specifies the type name identifier in XcodeML/Fortran.
 Refer to "\ref{sec:Thedatatypenameidentifier} The data type name identifier".}{O}
\XcodeMLAttrDef{scope}{text}
{Specifies one of
\newline
"{\tt local}" for a local variable,
\newline
"{\tt global}" for a global variable,
\newline
"{\tt param}" for a dummy argument.
\newline
Note: "{\tt global}" is not used in XcodeML/Fortran, however.}{R}
\end{XcodeMLAttributes}


\subsubsection{ {\tt FmemberRef} element}

The {\tt FmemberRef} element represents a reference to a structure component.

%\subsubsection*{Contents model}
%{\tt
%(varRef)
%}
\XcodeMLContentsModel{ (varRef) }

\begin{XcodeMLChildElements}
\XcodeMLElementDef{varRef}
{Specifies the reference to the structure.}{R}
\end{XcodeMLChildElements}

\begin{XcodeMLAttributes}
\XcodeMLAttrDef{type}{text}
{Specifies the type name identifier in XcodeML/Fortran.
 Refer to "\ref{sec:Thedatatypenameidentifier} The data type name identifier".}{O}
\XcodeMLAttrDef{member}{text}
{Specifies the name of the component.}{R}
\XcodeMLAttrDef{is\_complex\_part}{bool}
{{\tt true} if refering to the {\tt complex} type partially. In this case, the member name(the value of the {\tt member} attribute) is either "{\tt RE}" or "{\tt IM}".}{O}
\end{XcodeMLAttributes}


\subsubsection{ {\tt FcoArrayRef} element}

The {\tt FcoArrayRef} element represents a reference to a coarray.

%\subsubsection*{Contents model}
%{\tt
%(varRef, (arrayIndex)+)
%}
\XcodeMLContentsModel{ (varRef, (arrayIndex)+) }

\begin{XcodeMLChildElements}
\XcodeMLElementDef{varRef}
{Specifies the coindexed named object.}{R}
\XcodeMLElementDef{arrayIndex}
{Specifies the image selector.}{R}
\end{XcodeMLChildElements}

\begin{XcodeMLAttributes}
\XcodeMLAttrDef{type}{text}
{Specifies the type name identifier corresponding to the referenced coarray in XcodeML/Fortran.
 Refer to "\ref{sec:Thedatatypenameidentifier} The data type name identifier".}{O}
\end{XcodeMLAttributes}


\subsubsection{ {\tt FarrayRef} element}

The {\tt FarrayRef} element represents a reference to an array section or an array element.
In the case representing a reference to an array element, specifies all of the index
elements by the {\tt arrayIndex} element.

%\subsubsection*{Contents model}
%{\tt
%(varRef, (arrayIndex | indexRange | FarrayConstructor | FarrayRef)*)
%}
\XcodeMLContentsModel{ (varRef, (arrayIndex | indexRange | FarrayConstructor | FarrayRef)*) }

\begin{XcodeMLChildElements}
\XcodeMLElementDef{varRef}
{Specifies the reference to the array.}{R}
\XcodeMLElementDef{arrayIndex}
{Specified if the element list of the array section is expressed by the subscripts.
 Can be repeated the number of the dimension times together with other elements.}{O}
\XcodeMLElementDef{indexRange}
{Specified if the element list of the array section is expressed by the upper and lower bounds.
 Can be repeated the number of the dimension times together with other elements.}{O}
\XcodeMLElementDef{FarrayConstructor}
{Specified if the element list of the array section is expressed by the vector subscript by the array constructor.
 Can be repeated the number of the dimension times together with other elements.}{O}
\XcodeMLElementDef{FarrayRef}
{Specified if the element list of the array section is expressed by the vector subscript by the array section.
 Can be repeated the number of the dimension times together with other elements.}{O}
\end{XcodeMLChildElements}

\begin{XcodeMLAttributes}
\XcodeMLAttrDef{type}{text}
{Specifies the type name identifier in XcodeML/Fortran.
 Refer to "\ref{sec:Thedatatypenameidentifier} The data type name identifier".}{O}
\end{XcodeMLAttributes}


\subsubsection{ {\tt FcharacterRef} element}

The {\tt FcharacterRef} element represents a reference to a character substring.

%\subsubsection*{Contents model}
%{\tt
%(varRef, indexRange?)
%}
\XcodeMLContentsModel{ (varRef, indexRange?) }

\begin{XcodeMLChildElements}
\XcodeMLElementDef{varRef}
{Specifies the reference to the character variable.}{R}
\XcodeMLElementDef{indexRange}
{Specifies the element list of the character string by the upper and lower bounds.}{O}
\end{XcodeMLChildElements}

\begin{XcodeMLAttributes}
\XcodeMLAttrDef{type}{text}
{Specifies the type name identifier in XcodeML/Fortran.
 Refer to "\ref{sec:Thedatatypenameidentifier} The data type name identifier".}{O}
\end{XcodeMLAttributes}


\subsubsection{ {\tt varRef} element}

The {\tt varRef} element represents a reference to a variable.

%\subsubsection*{Contents model}
%{\tt
%(Var | FmemberRef | FarrayRef | FcharacterRef | FcoArrayRef)
%}
\XcodeMLContentsModel{ (Var | FmemberRef | FarrayRef | FcharacterRef | FcoArrayRef) }

\begin{XcodeMLChildElements}
\XcodeMLElementDef{Var}
{Specifid if the element references the variable.}{R}
\XcodeMLElementDef{FmemberRef}
{Specifid if the element references the structure component.}{R}
\XcodeMLElementDef{FarrayRef}
{Specifid if the element references the array section of the array.}{R}
\XcodeMLElementDef{FcharacterRef}
{Specifid if the element references the character substring of the character string.}{R}
\XcodeMLElementDef{FcoArrayRef}
{Specifid if the element references the coarray.}{R}
\end{XcodeMLChildElements}

\begin{XcodeMLAttributes}
\XcodeMLAttrDef{type}{text}
{Specifies the type name identifier in XcodeML/Fortran.
 Refer to "\ref{sec:Thedatatypenameidentifier} The data type name identifier".}{O}
\end{XcodeMLAttributes}


\subsection{Function call}

\subsubsection{ {\tt functionCall} element}

The {\tt functionCall} element represents a function / subroutine call.

%\subsubsection*{Contents model}
%{\tt
%((name|FmemberRef), arguments?)
%}
\XcodeMLContentsModel{ ((name|FmemberRef), arguments?) }

\begin{XcodeMLChildElements}
\XcodeMLElementDef{name}
{Specifies the name of the function or subroutine.}{R}
\XcodeMLElementDef{FmemberRef}
{Specifies the name of the function or subroutine
 (in the case of the component of the structure or the type bound procedure).}{R}
\XcodeMLElementDef{arguments}
{Specifies the expressions of the arguments.}{O}
\end{XcodeMLChildElements}

\begin{XcodeMLAttributes}
\XcodeMLAttrDef{type}{text}
{Specifies the type name identifier in XcodeML/Fortran.
 Refer to "\ref{sec:Thedatatypenameidentifier} The data type name identifier".}{O}
\XcodeMLAttrDef{is\_intrinsic}{bool}
{Specifies if the intrinsic function / subroutine call.}{O}
\end{XcodeMLAttributes}


\subsubsection{ {\tt arguments} element}

%\subsubsection*{Contents model}
%{\tt
%(exprModel)*
%}
\XcodeMLContentsModel{ (exprModel)* }

\begin{XcodeMLChildElements}
\XcodeMLElementDef{exprModel}
{Specifies the expressions of the arguments. Refer to "\ref{sec:exprModel} exprModel".}{O}
\end{XcodeMLChildElements}

\begin{XcodeMLAttributes}
\XcodeMLAttrDef{-}{-}
{-}{-}
\end{XcodeMLAttributes}


\subsection{Binary operation element}

The elements representing the binary operators are described below.
\newline

\begin{XcodeMLOperations}
\XcodeMLOperation{plusExpr}{+}{addition}
\XcodeMLOperation{minusExpr}{-}{subtraction}
\XcodeMLOperation{mulExpr}{*}{multiplication}
\XcodeMLOperation{divExpr}{/}{division}
\XcodeMLOperation{FpowerExpr}{**}{exponentiation}
\XcodeMLOperation{FconcatExpr}{//}{concatenation}
\XcodeMLOperation{logEQExpr}{==
\newline
.EQ.}{equal}
\XcodeMLOperation{logNEQExpr}{/=
\newline
.NE.}{not equal}
\XcodeMLOperation{logGEExpr}{\textgreater=
\newline
.GE.}{greater than or equal}
\XcodeMLOperation{logGTExpr}{\textgreater
\newline
.GT.}{greater than}
\XcodeMLOperation{logLEExpr}{\textless=
\newline
.LE.}{less than or equal}
\XcodeMLOperation{logLTExpr}{\textless
\newline
.LT.}{less than}
\XcodeMLOperation{logAndExpr}{.AND.}{logical intersection}
\XcodeMLOperation{logOrExpr}{.OR.}{logical union}
\XcodeMLOperation{logEQVExpr}{.EQV.}{logical equivalence}
\XcodeMLOperation{logNEQVExpr}{.NEQV.}{logical non-equivalence}
\XcodeMLOperation{userBinaryExpr}{arbitrary}{depends on the interface.}
\end{XcodeMLOperations}

%\subsubsection*{Contents model}
%{\tt
%(exprModel, exprModel)
%}
\XcodeMLContentsModel{ (exprModel, exprModel) }

\begin{XcodeMLChildElements}
\XcodeMLElementDef{exprModel}
{Specifies the left-hand expression as the first operand, the right-hand expression as the second operand. Refer to "\ref{sec:exprModel} exprModel".}{R}
\end{XcodeMLChildElements}

\begin{XcodeMLAttributes}
\XcodeMLAttrDef{type}{text}
{Specifies the type name identifier in XcodeML/Fortran.
 Refer to "\ref{sec:Thedatatypenameidentifier} The data type name identifier".}{O}
\end{XcodeMLAttributes}


\subsection{Unary operation element}

The elements representing the unary operators are described below.
No element is prepared for the unary operator '+' because it is omitted by the canonicalization.
\newline

\begin{XcodeMLOperations}
\XcodeMLOperation{unaryMinusExpr}{-}{sign inversion}
\XcodeMLOperation{logNotExpr}{.NOT.}{logical complement}
\XcodeMLOperation{userUnaryExpr}{arbitrary}{depends on the interface.}
\end{XcodeMLOperations}

%\subsubsection*{Contents model}
%{\tt
%(exprModel)
%}
\XcodeMLContentsModel{ (exprModel) }

\begin{XcodeMLChildElements}
\XcodeMLElementDef{exprModel}
{Specifies the operand expression. Refer to "\ref{sec:exprModel} exprModel.}{R}
\end{XcodeMLChildElements}

\begin{XcodeMLAttributes}
\XcodeMLAttrDef{type}{text}
{Specifies the type name identifier in XcodeML/Fortran.
 Refer to "\ref{sec:Thedatatypenameidentifier} The data type name identifier".}{O}
\end{XcodeMLAttributes}


