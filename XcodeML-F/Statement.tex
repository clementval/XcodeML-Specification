\section{Statement element}

This is an XML element which corresponds to the syntax element in the Fortran 90 statement.
The line number is attached to each element as an attribute, which can be used to extract file information and the line number where the particular statement is found.

\subsection{ {\tt FassignStatement} element}

The {\tt FassignStatement} element represents an assignment statement.

%\subsubsection*{Contents model}
%{\tt
%(lValueModel, exprModel)
%}
\XcodeMLContentsModel{ (lValueModel, exprModel) }

\begin{XcodeMLChildElements}
\XcodeMLElementDef{lValueModel}
{Specifies the left-hand side expression. Refer to "\ref{sec:lValueModel} lValueModel".}{R}
\XcodeMLElementDef{exprModel}
{Specifies the right-hand side expression. Refer to "\ref{sec:exprModel} exprModel".}{R}
\end{XcodeMLChildElements}

\begin{XcodeMLAttributes}
\XcodeMLAttrDef{common attributes}{-}
{Refer to "\ref{sec:Commonattributesofdefinition} Common attributes of the definition / declaration / statement element".}{-}
\end{XcodeMLAttributes}


\subsection{ {\tt exprStatement} element}

The {\tt exprStatement} element represents a statement expressed by an expression.

%\subsubsection*{Contents model}
%{\tt
%(exprModel)
%}
\XcodeMLContentsModel{ (exprModel) }

\begin{XcodeMLChildElements}
\XcodeMLElementDef{exprModel}
{Specifies the expression. Refer to "\ref{sec:exprModel} exprModel"}{R}
\end{XcodeMLChildElements}

\begin{XcodeMLAttributes}
\XcodeMLAttrDef{common attributes}{-}
{Refer to "\ref{sec:Commonattributesofdefinition} Common attributes of the definition / declaration / statement element".}{-}
\end{XcodeMLAttributes}


\subsection{ {\tt FpointerAssignStatement} element}

The {\tt FpointerAssignStatement} element represents a pointer assignment statement.

%\subsubsection*{Contents model}
%{\tt
%(lValueMode, exprModel)
%}
\XcodeMLContentsModel{ (lValueMode, exprModel) }

\begin{XcodeMLChildElements}
\XcodeMLElementDef{lValueModel}
{Specifies the left-hand side expression. Refer to "\ref{sec:lValueModel} lValueModel".}{R}
\XcodeMLElementDef{exprModel}
{Specifies the right-hand side expression. Refer to "\ref{sec:exprModel} exprModel".}{R}
\end{XcodeMLChildElements}

\begin{XcodeMLAttributes}
\XcodeMLAttrDef{common attributes}{-}
{Refer to "\ref{sec:Commonattributesofdefinition} Common attributes of the definition / declaration / statement element".}{-}
\end{XcodeMLAttributes}


\subsection{ {\tt FifStatement} element}

The {\tt FifStatement} element represents an {\tt if} statement.

%\subsubsection*{Contents model}
%{\tt
%(condition, then, else?)
%}
\XcodeMLContentsModel{ (condition, then, else?) }

\begin{XcodeMLChildElements}
\XcodeMLElementDef{condition}
{Specifies the conditional expression.}{R}
\XcodeMLElementDef{then}
{Specifies the statement to execute if the condition falls true.}{R}
\XcodeMLElementDef{else}
{Specifies the statement to execute if the condition falls false.}{O}
\end{XcodeMLChildElements}

\begin{XcodeMLAttributes}
\XcodeMLAttrDef{common attributes}{-}
{Refer to "\ref{sec:Commonattributesofdefinition} Common attributes of the definition / declaration / statement element".}{-}
\XcodeMLAttrDef{construct\_name}{text}
{Specifies the construct name.}{O}
\end{XcodeMLAttributes}


\subsection{ {\tt FdoStatement} element}

The {\tt FdoStatement} element represents a {\tt do} statement.
The labeled {\tt do} construct must be replaced by an equivalent {\tt do} statement with the label eliminated.

%\subsubsection*{Contents model}
%{\tt
%(Var?, indexRange?, body?)
%}
\XcodeMLContentsModel{ (Var?, indexRange?, body?) }

\begin{XcodeMLChildElements}
\XcodeMLElementDef{Var}
{Specifies the index variable.}{O}
\XcodeMLElementDef{indexRange}
{Specifies the range of of the value of the index variable.}{O}
\XcodeMLElementDef{body}
{Specifies the statements included in the {\tt do} statement.}{O}
\end{XcodeMLChildElements}

\begin{XcodeMLAttributes}
\XcodeMLAttrDef{common attributes}{-}
{Refer to "\ref{sec:Commonattributesofdefinition} Common attributes of the definition / declaration / statement element".}{-}
\XcodeMLAttrDef{construct\_name}{text}
{Specifies the construct name.}{O}
\end{XcodeMLAttributes}


\subsection{ {\tt FdoWhileStatement} element}

The {\tt FdoWhileStatement} element represents a {\tt do} statement with {\tt while} used as the controlling expression.
The labelled {\tt do while} construct must be replaced by an equivalent {\tt do while} statement with the label eliminated.

%\subsubsection*{Contents model}
%{\tt
%(condition, body?)
%}
\XcodeMLContentsModel{ (condition, body?) }

\begin{XcodeMLChildElements}
\XcodeMLElementDef{condition}
{Specifies the conditional expression.}{R}
\XcodeMLElementDef{body}
{Specifies the statements included in the {\tt do while} statement.}{O}
\end{XcodeMLChildElements}

\begin{XcodeMLAttributes}
\XcodeMLAttrDef{common attributes}{-}
{Refer to "\ref{sec:Commonattributesofdefinition} Common attributes of the definition / declaration / statement element".}{-}
\XcodeMLAttrDef{construct\_name}{text}
{Specifies the construct name.}{O}
\end{XcodeMLAttributes}


\subsection{ {\tt continueStatement} element}

The {\tt continueStatement} element represents a {\tt continue} statement.

%\subsubsection*{Contents model}
%{\tt
%empty
%}
\XcodeMLContentsModel{ empty }

\begin{XcodeMLChildElements}
\XcodeMLElementDef{-}
{-}{-}
\end{XcodeMLChildElements}

\begin{XcodeMLAttributes}
\XcodeMLAttrDef{common attributes}{-}
{Refer to "\ref{sec:Commonattributesofdefinition} Common attributes of the definition / declaration / statement element".}{-}
\end{XcodeMLAttributes}


\subsection{ {\tt FcycleStatement} element}

The {\tt FcycleStatement} element represents a {\tt cycle} statement.

%\subsubsection*{Contents model}
%{\tt
%empty
%}
\XcodeMLContentsModel{ empty }

\begin{XcodeMLChildElements}
\XcodeMLElementDef{-}
{-}{-}
\end{XcodeMLChildElements}

\begin{XcodeMLAttributes}
\XcodeMLAttrDef{common attributes}{-}
{Refer to "\ref{sec:Commonattributesofdefinition} Common attributes of the definition / declaration / statement element".}{-}
\XcodeMLAttrDef{construct\_name}{text}
{Specifies the construct name.}{O}
\end{XcodeMLAttributes}


\subsection{ {\tt FexitStatement} element}

The {\tt FexitStatement} element represents an {\tt exit} statement.

%\subsubsection*{Contents model}
%{\tt
%empty
%}
\XcodeMLContentsModel{ empty }

\begin{XcodeMLChildElements}
\XcodeMLElementDef{-}
{-}{-}
\end{XcodeMLChildElements}

\begin{XcodeMLAttributes}
\XcodeMLAttrDef{common attributes}{-}
{Refer to "\ref{sec:Commonattributesofdefinition} Common attributes of the definition / declaration / statement element".}{-}
\XcodeMLAttrDef{construct\_name}{text}
{Specifies the construct name.}{O}
\end{XcodeMLAttributes}


\subsection{ {\tt FreturnStatement} element}

The {\tt FreturnStatement} element represents a {\tt return} statement.

%\subsubsection*{Contents model}
%{\tt
%empty
%}
\XcodeMLContentsModel{ empty }

\begin{XcodeMLChildElements}
\XcodeMLElementDef{-}
{-}{-}
\end{XcodeMLChildElements}

\begin{XcodeMLAttributes}
\XcodeMLAttrDef{common attributes}{-}
{Refer to "\ref{sec:Commonattributesofdefinition} Common attributes of the definition / declaration / statement element".}{-}
\end{XcodeMLAttributes}


\subsection{ {\tt gotoStatement} element}

The {\tt gotoStatement} element represents a {\tt go to} statement.

%\subsubsection*{Contents model}
%{\tt
%((params, value)?)
%}
\XcodeMLContentsModel{ ((params, value)?) }

\begin{XcodeMLChildElements}
\XcodeMLElementDef{params}
{Specifies the statement label list of the computed {\tt go to} statement.
 Neglected if the {\tt label\_name} attribute is specified.}{O}
\XcodeMLElementDef{value}
{Specifies the expression of the computed {\tt go to} statement.
 Neglected if the {\tt label\_name} attribute is specified.}{O}
\end{XcodeMLChildElements}

\begin{XcodeMLAttributes}
\XcodeMLAttrDef{common attributes}{-}
{Refer to "\ref{sec:Commonattributesofdefinition} Common attributes of the definition / declaration / statement element".}{-}
\XcodeMLAttrDef{label\_name}{text}
{Specifies the statement label.}{O}
\end{XcodeMLAttributes}


\subsection{ {\tt statementLabel} element}

The {\tt statementLabel} element represents a statement label of the following statement.

%\subsubsection*{Contents model}
%{\tt
%empty
%}
\XcodeMLContentsModel{ empty }

\begin{XcodeMLChildElements}
\XcodeMLElementDef{-}
{-}{-}
\end{XcodeMLChildElements}

\begin{XcodeMLAttributes}
\XcodeMLAttrDef{common attributes}{-}
{Refer to "\ref{sec:Commonattributesofdefinition} Common attributes of the definition / declaration / statement element".}{-}
\XcodeMLAttrDef{label\_name}{text}
{Specifies the statement label.}{R}
\end{XcodeMLAttributes}


\subsection{ {\tt FselectCaseStatement} element}

The {\tt FselectCaseStatement} element represents a {\tt select case} construct.

%\subsubsection*{Contents model}
%{\tt
%(value, FcaseLabel*)
%}
\XcodeMLContentsModel{ (value, FcaseLabel*) }

\begin{XcodeMLChildElements}
\XcodeMLElementDef{value}
{Specifies the value to select.}{R}
\XcodeMLElementDef{FcaseLabel}
{Specifies the {\tt case} statements included in the element.}{O}
\end{XcodeMLChildElements}

\begin{XcodeMLAttributes}
\XcodeMLAttrDef{common attributes}{-}
{Refer to "\ref{sec:Commonattributesofdefinition} Common attributes of the definition / declaration / statement element".}{-}
\XcodeMLAttrDef{construct\_name}{text}
{Specifies the construct name.}{O}
\end{XcodeMLAttributes}


\subsection{ {\tt FcaseLabel} element}

The {\tt FcaseLabel} element represents a {\tt case} statement in a {\tt select case} construct.
If a {\tt FcaseLabel} has neither a {\tt value} element nor an {\tt indexRange} element, it is assume to be the {\tt case default} statement.

%\subsubsection*{Contents model}
%{\tt
%((value | indexRange)*, body)
%}
\XcodeMLContentsModel{ ((value | indexRange)*, body) }

\begin{XcodeMLChildElements}
\XcodeMLElementDef{value}
{Specifies the value of the selector.}{O}
\XcodeMLElementDef{indexRange}
{Specifies the range of the selector.}{O}
\XcodeMLElementDef{body}
{Specifies the statements included in the element.}{R}
\end{XcodeMLChildElements}

\begin{XcodeMLAttributes}
\XcodeMLAttrDef{common attributes}{-}
{Refer to "\ref{sec:Commonattributesofdefinition} Common attributes of the definition / declaration / statement element".}{-}
\XcodeMLAttrDef{construct\_name}{text}
{Specifies the construct name of the corresponding {\tt select case} statement.}{O}
\end{XcodeMLAttributes}


\subsection{ {\tt FwhereStatement} element}

The {\tt FwhereStatement} element represents a {\tt where} statement.

%\subsubsection*{Contents model}
%{\tt
%(condition, then, else?)
%}
\XcodeMLContentsModel{ (condition, then, else?) }

\begin{XcodeMLChildElements}
\XcodeMLElementDef{condition}
{Specifies the conditional expression.}{R}
\XcodeMLElementDef{then}
{Specifies the statement to execute if the condition falls true.}{R}
\XcodeMLElementDef{else}
{Specifies the statement to execute if the condition falls false.}{O}
\end{XcodeMLChildElements}

\begin{XcodeMLAttributes}
\XcodeMLAttrDef{common attributes}{-}
{Refer to "\ref{sec:Commonattributesofdefinition} Common attributes of the definition / declaration / statement element".}{-}
\end{XcodeMLAttributes}


\subsection{ {\tt FstopStatement} element}

The {\tt FstopStatement} element represents a {\tt stop} statement.

%\subsubsection*{Contents model}
%{\tt
%empty
%}
\XcodeMLContentsModel{ empty }

\begin{XcodeMLChildElements}
\XcodeMLElementDef{-}
{-}{-}
\end{XcodeMLChildElements}

\begin{XcodeMLAttributes}
\XcodeMLAttrDef{common attributes}{-}
{Refer to "\ref{sec:Commonattributesofdefinition} Common attributes of the definition / declaration / statement element".}{-}
\XcodeMLAttrDef{code}{text}
{Specifies a stop code consisting of a string of up to five digits.}{O}
\XcodeMLAttrDef{message}{text}
{Specifies a stop code consisting of a default character constant.
 Neglected if the code attribute is specified.}{O}
\end{XcodeMLAttributes}


\subsection{Input/Output element}

\subsubsection{ {\tt FreadStatement, FwriteStatement} element}

The {\tt FreadStatement} element and the {\tt FwriteStatement} element correspond to the {\tt read} statement and the {\tt write} statement respectively.

%\subsubsection*{Contents model}
%{\tt
%(namedValueList, valueList)
%}
\XcodeMLContentsModel{ (namedValueList, valueList) }

\begin{XcodeMLChildElements}
\XcodeMLElementDef{namedValueList}
{Specifies the control list.}{R}
\XcodeMLElementDef{valueList}
{Specifies the input or output list.}{R}
\end{XcodeMLChildElements}

\begin{XcodeMLAttributes}
\XcodeMLAttrDef{common attributes}{-}
{Refer to "\ref{sec:Commonattributesofdefinition} Common attributes of the definition / declaration / statement element".}{-}
\end{XcodeMLAttributes}

\begin{XcodeMLControlList}
\XcodeMLSpecifier{unit}
{Specifies either '{\tt *}', the scalar default integer expression specifying the external file unit or the scalar default character variable specifying the internal file unit.}{R}
\XcodeMLSpecifier{fmt}
{Specifies the format specifier, either '{\tt *}', the scalar default character expression, the statement label or the scalar default integer variable.}{R}
\XcodeMLSpecifier{nml}
{Specifies the name of the variable group.}{O}
\XcodeMLSpecifier{rec}
{Specifies the name of the scalar default integer variable.}{O}
\XcodeMLSpecifier{iostat}
{Specifies the name of the scalar default integer variable.}{O}
\XcodeMLSpecifier{err}
{Specifies the statement label.}{O}
\XcodeMLSpecifier{end}
{Specifies the statement label.}{O}
\XcodeMLSpecifier{advance}
{Specifies the scalar default character expression.}{O}
\XcodeMLSpecifier{size}
{Specifies the name of the scalar default integer variable.}{O}
\XcodeMLSpecifier{eor}
{Specifies the statement label.}{O}
\XcodeMLSpecifier{pos}
{Specifies the scalar integer expression.}{O}
\XcodeMLSpecifier{iomsg}
{Specifies the scalar default character expression.}{O}
\XcodeMLSpecifier{blank}
{Specifies the scalar default character expression.}{O}
\XcodeMLSpecifier{pad}
{Specifies the scalar default character expression.}{O}
\XcodeMLSpecifier{decimal}
{Specifies the scalar default character expression.}{O}
\XcodeMLSpecifier{delim}
{Specifies the scalar default character expression.}{O}
\XcodeMLSpecifier{round}
{Specifies the scalar default character expression.}{O}
\XcodeMLSpecifier{sign}
{Specifies the scalar default character expression.}{O}
\XcodeMLSpecifier{asynchronous}
{Specifies the scalar default character expression.}{O}
\XcodeMLSpecifier{id}
{Specifies the name of the variable.}{O}
\end{XcodeMLControlList}


\subsubsection{ {\tt FprintStatement} element}

The {\tt FprintStatement} element corresponds to the {\tt print} statement.

%\subsubsection*{Contents model}
%{\tt
%(valueList)
%}
\XcodeMLContentsModel{ (valueList) }

\begin{XcodeMLChildElements}
\XcodeMLElementDef{valueList}
{Specifies the output list.}{R}
\end{XcodeMLChildElements}

\begin{XcodeMLAttributes}
\XcodeMLAttrDef{common attributes}{-}
{Refer to "\ref{sec:Commonattributesofdefinition} Common attributes of the definition / declaration / statement element".}{-}
\XcodeMLAttrDef{format}{text}
{Specifies the fortmat specifier.}{R}
\end{XcodeMLAttributes}


\subsubsection{ {\tt FrewindStatement, FendFileStatement, FbackspaceStatement} element}

The {\tt FrewindStatement} element, the {\tt FendFileStatement} element and the {\tt FbackspaceStatement} element
correspond to the {\tt rewind} statement, the {\tt end file} statement and the {\tt backspace} statement respectively.

%\subsubsection*{Contents model}
%{\tt
%(namedValueList)
%}
\XcodeMLContentsModel{ (namedValueList) }

\begin{XcodeMLChildElements}
\XcodeMLElementDef{namedValueList}
{Specifies the control list.}{R}
\end{XcodeMLChildElements}

\begin{XcodeMLAttributes}
\XcodeMLAttrDef{common attributes}{-}
{Refer to "\ref{sec:Commonattributesofdefinition} Common attributes of the definition / declaration / statement element".}{-}
\end{XcodeMLAttributes}

\begin{XcodeMLControlList}
\XcodeMLSpecifier{unit}
{Specifies the scalar default integer expression specifying the external file unit.}{R}
\XcodeMLSpecifier{iostat}
{Specifies the name of the scalar default integer variable.}{O}
\XcodeMLSpecifier{err}
{Specifies the statement label.}{O}
\XcodeMLSpecifier{iomsg}
{Specifies the scalar default character expression.}{O}
\end{XcodeMLControlList}


\subsubsection{ {\tt FopenStatement} element}

The {\tt FopenStatement} element corresponds to the {\tt open} statement.

%\subsubsection*{Contents model}
%{\tt
%(valueList)
%}
\XcodeMLContentsModel{ (valueList) }

\begin{XcodeMLChildElements}
\XcodeMLElementDef{valueList}
{Specifies the output list.}{R}
\end{XcodeMLChildElements}

\begin{XcodeMLAttributes}
\XcodeMLAttrDef{common attributes}{-}
{Refer to "\ref{sec:Commonattributesofdefinition} Common attributes of the definition / declaration / statement element".}{-}
\end{XcodeMLAttributes}

\begin{XcodeMLControlList}
\XcodeMLSpecifier{unit}
{Specifies the scalar default integer expression specifying the external file unit.}{R}
\XcodeMLSpecifier{iostat}
{Specifies the name of the scalar default integer variable.}{O}
\XcodeMLSpecifier{err}
{Specifies the statement label.}{O}
\XcodeMLSpecifier{file}
{Specifies the name of the scalar default character variable containing the file name.}{O}
\XcodeMLSpecifier{status}
{Specifies the name of the scalar default character variable.}{O}
\XcodeMLSpecifier{access}
{Specifies the name of the scalar default character variable.}{O}
\XcodeMLSpecifier{form}
{Specifies the name of the scalar default character variable.}{O}
\XcodeMLSpecifier{recl}
{Specifies the name of the scalar default integer variable.}{O}
\XcodeMLSpecifier{blank}
{Specifies the name of the scalar default character variable.}{O}
\XcodeMLSpecifier{position}
{Specifies the name of the scalar default character variable.}{O}
\XcodeMLSpecifier{action}
{Specifies the name of the scalar default character variable.}{O}
\XcodeMLSpecifier{delim}
{Specifies the name of the scalar default character variable.}{O}
\XcodeMLSpecifier{pad}
{Specifies the name of the scalar default character variable.}{O}
\XcodeMLSpecifier{newunit}
{Specifies the name of the scalar default integer variable.}{O}
\XcodeMLSpecifier{iomsg}
{Specifies the scalar default character expression.}{O}
\XcodeMLSpecifier{decimal}
{Specifies the scalar default character expression.}{O}
\XcodeMLSpecifier{delim}
{Specifies the scalar default character expression.}{O}
\XcodeMLSpecifier{encoding}
{Specifies the scalar default character expression.}{O}
\XcodeMLSpecifier{round}
{Specifies the scalar default character expression.}{O}
\XcodeMLSpecifier{sign}
{Specifies the scalar default character expression.}{O}
\XcodeMLSpecifier{asynchronous}
{Specifies the scalar default character expression.}{O}
\end{XcodeMLControlList}


\subsubsection{ {\tt FcloseStatement} element}

The {\tt FcloseStatement} element corresponds to the {\tt close} statement.

%\subsubsection*{Contents model}
%{\tt
%(valueList)
%}
\XcodeMLContentsModel{ (valueList) }

\begin{XcodeMLChildElements}
\XcodeMLElementDef{valueList}
{Specifies the output list.}{R}
\end{XcodeMLChildElements}

\begin{XcodeMLAttributes}
\XcodeMLAttrDef{common attributes}{-}
{Refer to "\ref{sec:Commonattributesofdefinition} Common attributes of the definition / declaration / statement element".}{-}
\end{XcodeMLAttributes}

\begin{XcodeMLControlList}
\XcodeMLSpecifier{unit}
{Specifies the scalar default integer expression specifying the external file unit.}{R}
\XcodeMLSpecifier{iostat}
{Specifies the name of the scalar default integer variable.}{O}
\XcodeMLSpecifier{err}
{Specifies the statement label.}{O}
\XcodeMLSpecifier{status}
{Specifies the name of the scalar default character variable.}{O}
\XcodeMLSpecifier{iomsg}
{Specifies the scalar default character expression.}{O}
\end{XcodeMLControlList}


\subsubsection{ {\tt FinquireStatement} element}

The {\tt FinquireStatement} element corresponds to the {\tt inquire} statement.

%\subsubsection*{Contents model}
%{\tt
%(namedValueList, valueList)
%}
\XcodeMLContentsModel{ (namedValueList, valueList) }

\begin{XcodeMLChildElements}
\XcodeMLElementDef{namedValueList}
{Specifies the control list.}{R}
\XcodeMLElementDef{valueList}
{Specifies the input or output list.}{R}
\end{XcodeMLChildElements}

\begin{XcodeMLAttributes}
\XcodeMLAttrDef{common attributes}{-}
{Refer to "\ref{sec:Commonattributesofdefinition} Common attributes of the definition / declaration / statement element".}{-}
\end{XcodeMLAttributes}

\begin{XcodeMLControlList}
\XcodeMLSpecifier{iolength}
{Specifies the scalar default integer expression. If this is specifies, no other specifiers
 can be specified and the output list must be specified by the {\tt valueList} element.}{O}
\XcodeMLSpecifier{unit}
{Specifies the scalar default integer expression specifying the external file unit.}{O}
\XcodeMLSpecifier{file}
{Specifies the name of the scalar default character variable containing the file name.}{O}
\XcodeMLSpecifier{iostat}
{Specifies the name of the scalar default integer variable.}{O}
\XcodeMLSpecifier{err}
{Specifies the statement label.}{O}
\XcodeMLSpecifier{exist}
{Specifies the name of the scalar default logical variable.}{O}
\XcodeMLSpecifier{opened}
{Specifies the name of the scalar default logical variable.}{O}
\XcodeMLSpecifier{number}
{Specifies the name of the scalar default integer variable.}{O}
\XcodeMLSpecifier{named}
{Specifies the name of the scalar default logical variable.}{O}
\XcodeMLSpecifier{name}
{Specifies the name of the scalar default character variable.}{O}
\XcodeMLSpecifier{access}
{Specifies the name of the scalar default character variable.}{O}
\XcodeMLSpecifier{sequential}
{Specifies the name of the scalar default character variable.}{O}
\XcodeMLSpecifier{direct}
{Specifies the name of the scalar default character variable.}{O}
\XcodeMLSpecifier{form}
{Specifies the name of the scalar default character variable.}{O}
\XcodeMLSpecifier{formatted}
{Specifies the name of the scalar default character variable.}{O}
\XcodeMLSpecifier{unformatted}
{Specifies the name of the scalar default character variable.}{O}
\XcodeMLSpecifier{recl}
{Specifies the name of the scalar default integer variable.}{O}
\XcodeMLSpecifier{nextrecl}
{Specifies the name of the scalar default integer variable.}{O}
\XcodeMLSpecifier{blank}
{Specifies the name of the scalar default character variable.}{O}
\XcodeMLSpecifier{position}
{Specifies the name of the scalar default character variable.}{O}
\XcodeMLSpecifier{action}
{Specifies the name of the scalar default character variable.}{O}
\XcodeMLSpecifier{read}
{Specifies the name of the scalar default character variable.}{O}
\XcodeMLSpecifier{write}
{Specifies the name of the scalar default character variable.}{O}
\XcodeMLSpecifier{rewrite}
{Specifies the name of the scalar default character variable.}{O}
\XcodeMLSpecifier{delim}
{Specifies the name of the scalar default character variable.}{O}
\XcodeMLSpecifier{pad}
{Specifies the name of the scalar default character variable.}{O}
\XcodeMLSpecifier{pending}
{Specifies the name of the scalar default logical variable.}{O}
\XcodeMLSpecifier{pos}
{Specifies the scalar default integer expression.}{O}
\XcodeMLSpecifier{size}
{Specifies the scalar default integer expression.}{O}
\XcodeMLSpecifier{iomsg}
{Specifies the scalar default character expression.}{O}
\XcodeMLSpecifier{decimal}
{Specifies the scalar default character expression.}{O}
\XcodeMLSpecifier{delim}
{Specifies the scalar default character expression.}{O}
\XcodeMLSpecifier{encoding}
{Specifies the scalar default character expression.}{O}
\XcodeMLSpecifier{round}
{Specifies the scalar default character expression.}{O}
\XcodeMLSpecifier{sign}
{Specifies the scalar default character expression.}{O}
\XcodeMLSpecifier{stream}
{Specifies the scalar default character expression.}{O}
\XcodeMLSpecifier{asynchronous}
{Specifies the scalar default character expression.}{O}
\XcodeMLSpecifier{is}
{Specifies the scalar integer expression.}{O}
\end{XcodeMLControlList}


\subsubsection{ {\tt FwaitStatement} element}

The {\tt Fwait} element corresponds to the {\tt wait} statement.

%\subsubsection*{Contents model}
%{\tt
%empty
%}
\XcodeMLContentsModel{ empty }

\begin{XcodeMLChildElements}
\XcodeMLElementDef{-}
{-}{-}
\end{XcodeMLChildElements}

\begin{XcodeMLAttributes}
\XcodeMLAttrDef{common attributes}{-}
{Refer to "\ref{sec:Commonattributesofdefinition} Common attributes of the definition / declaration / statement element".}{-}
\end{XcodeMLAttributes}

\begin{XcodeMLControlList}
\XcodeMLSpecifier{unit}
{Specifies either '{\tt *}', the scalar default integer expression specifying the external file unit or the scalar default character variable specifying the internal file unit.}{R}
\XcodeMLSpecifier{err}
{Specifies the statement label.}{O}
\XcodeMLSpecifier{end}
{Specifies the statement label.}{O}
\XcodeMLSpecifier{eor}
{Specifies the statement label.}{O}
\XcodeMLSpecifier{iostat}
{Specifies the scalar default integer variable.}{O}
\XcodeMLSpecifier{iomsg}
{Specifies the scalar default character expression.}{O}
\XcodeMLSpecifier{id}
{Specifies the scalar default integer expression.}{O}
\end{XcodeMLControlList}

\subsubsection{ {\tt FflushStatement} element}

The {\tt FflushStatement} element corresponds to the {\tt flush} statement.

%\subsubsection*{Contents model}
%{\tt
%empty
%}
\XcodeMLContentsModel{ empty }

\begin{XcodeMLChildElements}
\XcodeMLElementDef{}
{-}{-}
\end{XcodeMLChildElements}

\begin{XcodeMLAttributes}
\XcodeMLAttrDef{common attributes}{-}
{Refer to "\ref{sec:Commonattributesofdefinition} Common attributes of the definition / declaration / statement element".}{-}
\end{XcodeMLAttributes}

\begin{XcodeMLControlList}
\XcodeMLSpecifier{unit}
{Specifies the scalar default integer expression specifying the external file unit.}{R}
\XcodeMLSpecifier{err}
{Specifies the statement label.}{O}
\XcodeMLSpecifier{iomsg}
{Specifies the scalar default character expression.}{O}
\XcodeMLSpecifier{iostat}
{Specifies the scalar default integer variable.}{O}
\end{XcodeMLControlList}


\subsection{ {\tt FformatDecl} element}

The {\tt FformatDecl} element represents a {\tt format} statement.

%\subsubsection*{Contents model}
%{\tt
%empty
%}
\XcodeMLContentsModel{ empty }

\begin{XcodeMLChildElements}
\XcodeMLElementDef{-}
{-}{-}
\end{XcodeMLChildElements}

\begin{XcodeMLAttributes}
\XcodeMLAttrDef{common attributes}{-}
{Refer to "\ref{sec:Commonattributesofdefinition} Common attributes of the definition / declaration / statement element".}{-}
\XcodeMLAttrDef{format}{text}
{Specifies the character string representing the format specification.}{R}
\end{XcodeMLAttributes}


\subsection{ {\tt FdataDecl} element}

The {\tt FdataDecl} element represents a {\tt data} statement.

%\subsubsection*{Contents model}
%{\tt
%((varList, valueList)+)
%}
\XcodeMLContentsModel{ ((varList, valueList)+) }

\begin{XcodeMLChildElements}
\XcodeMLElementDef{varList}
{Specifies the object list.}{R}
\XcodeMLElementDef{valueList}
{Specifies the value list.}{R}
\end{XcodeMLChildElements}

\begin{XcodeMLAttributes}
\XcodeMLAttrDef{common attributes}{-}
{Refer to "\ref{sec:Commonattributesofdefinition} Common attributes of the definition / declaration / statement element".}{-}
\end{XcodeMLAttributes}


\subsection{ {\tt FnamelistDecl} element}

The {\tt FnamelistDecl} element represents a {\tt namelist} statement.

%\subsubsection*{Contents model}
%{\tt
%(varList+)
%}
\XcodeMLContentsModel{ (varList+) }

\begin{XcodeMLChildElements}
\XcodeMLElementDef{varList}
{Specifies the name of the variable group and the list of the element variables.}{R}
\end{XcodeMLChildElements}

\begin{XcodeMLAttributes}
\XcodeMLAttrDef{common attributes}{-}
{Refer to "\ref{sec:Commonattributesofdefinition} Common attributes of the definition / declaration / statement element".}{-}
\end{XcodeMLAttributes}


\subsection{ {\tt FequivalenceDecl} element}

The {\tt FequivalenceDecl} element represents an {\tt equivalent} statement.

%\subsubsection*{Contents model}
%{\tt
%((varRef, varList)+)
%}
\XcodeMLContentsModel{ ((varRef, varList)+) }

\begin{XcodeMLChildElements}
\XcodeMLElementDef{varRef}
{Specifies the object. The object must be a variable, array element or substring.}{R}
\XcodeMLElementDef{varList}
{Specifies the object list.}{R}
\end{XcodeMLChildElements}

\begin{XcodeMLAttributes}
\XcodeMLAttrDef{common attributes}{-}
{Refer to "\ref{sec:Commonattributesofdefinition} Common attributes of the definition / declaration / statement element".}{-}
\end{XcodeMLAttributes}


\subsection{ {\tt FcommonDecl} element}

The {\tt FcommonDecl} element represents a {\tt common} statement.

%\subsubsection*{Contents model}
%{\tt
%(varList+)
%}
\XcodeMLContentsModel{ (varList+) }

\begin{XcodeMLChildElements}
\XcodeMLElementDef{varList}
{Specifies the name of the common block and the list of the variable names.}{R}
\end{XcodeMLChildElements}

\begin{XcodeMLAttributes}
\XcodeMLAttrDef{common attributes}{-}
{Refer to "\ref{sec:Commonattributesofdefinition} Common attributes of the definition / declaration / statement element".}{-}
\end{XcodeMLAttributes}


\subsection{ {\tt FentryDecl} element}

The {\tt FentryDecl} element represents an {\tt entry} statement.

%\subsubsection*{Contents model}
%{\tt
%(name, symbols?, params?)
%}
\XcodeMLContentsModel{ (name, symbols?, params?) }

\begin{XcodeMLChildElements}
\XcodeMLElementDef{name}
{Specifies the name of entry.}{R}
\XcodeMLElementDef{symbols}
{Specifies the symbols included in the element.}{O}
\XcodeMLElementDef{params}
{Specifies the dummy arguments.}{O}
\end{XcodeMLChildElements}

\begin{XcodeMLAttributes}
\XcodeMLAttrDef{common attributes}{-}
{Refer to "\ref{sec:Commonattributesofdefinition} Common attributes of the definition / declaration / statement element".}{-}
\end{XcodeMLAttributes}


\subsection{Allocation element}

The following elements define the allocate statements.

\subsubsection{ {\tt FallocateStatement} element}

The {\tt FallocateStatement} element represents an {\tt allocate} statement.

%\subsubsection*{Contents model}
%{\tt
%(alloc+, alloc\_opt*)
%}
\XcodeMLContentsModel{ (alloc+, alloc\_opt*) }

\begin{XcodeMLChildElements}
\XcodeMLElementDef{alloc}
{Specifies the allocation list.}{R}
\XcodeMLElementDef{alloc\_opt}
{Specifies the allocation option.}{O}
\end{XcodeMLChildElements}

\begin{XcodeMLAttributes}
\XcodeMLAttrDef{common attributes}{-}
{Refer to "\ref{sec:Commonattributesofdefinition} Common attributes of the definition / declaration / statement element".}{-}
\XcodeMLAttrDef{stat\_name}{text}
{Specifies the name of status variable.}{O}
\XcodeMLAttrDef{type}{text}
{Specifies the type identifier if the type is specified.}{O}
\end{XcodeMLAttributes}

\subsubsection*{Example}

The {\tt FallocateStatement} element for the {\tt allocate} statement below is as follows:
\vspace{2mm}

\begin{Fexample2008}
Class(t),pointer :: x(:)
Class(*),pointer :: a(:),b(:)
allocate(et::x(100),a(100))
allocate(b,source=a,stat=s)
\end{Fexample2008}
\vspace{1mm}

\begin{XcodeMLFExample}
<FallocateStatement type="ID_OF_ET">
  <alloc>
    <Var>x</Var>
    <arrayIndex>
      <FintConstant>100</FintConstant>
    </arrayIndex>
  </alloc>
  <alloc>
     <Var>a</Var>
     <arrayIndex>
       <FintConstant>100</FintConstant>
     </arrayIndex>
   </alloc>
</FallocateStatement>
<FallocateStatement>
  <alloc>
    <Var>b</Var>
  </alloc>
  <alloc\_opt kind="source">
     <Var>a</Var>
  </alloc\_opt>
  <alloc\_opt kind="stat">
     <Var>s</Var>
  </alloc\_opt>
</FallocateStatement>
\end{XcodeMLFExample}


\subsubsection{ {\tt FdeallocateStatement} element}

The {\tt FdeallocateStatement} element represents a {\tt deallocate} statement.

%\subsubsection*{Contents model}
%{\tt
%(alloc+, alloc\_opt*)
%}
\XcodeMLContentsModel{ (alloc+, alloc\_opt*) }

\begin{XcodeMLChildElements}
\XcodeMLElementDef{alloc}
{Specifies the name of the allocate object list.}{R}
\XcodeMLElementDef{alloc\_opt}
{Specifies the allocation option.}{O}
\end{XcodeMLChildElements}

\begin{XcodeMLAttributes}
\XcodeMLAttrDef{common attributes}{-}
{Refer to "\ref{sec:Commonattributesofdefinition} Common attributes of the definition / declaration / statement element".}{-}
\XcodeMLAttrDef{stat\_name}{text}
{Specifies the name of status variable.}{O}
\end{XcodeMLAttributes}


\subsubsection{ {\tt FnullifyStatement} element}

The {\tt FnullifyStatement} element represents a {\tt nullify} statement.

%\subsubsection*{Contents model}
%{\tt
%(alloc+)
%}
\XcodeMLContentsModel{ (alloc+) }

\begin{XcodeMLChildElements}
\XcodeMLElementDef{alloc}
{Specifies the pointer object list.}{R}
\end{XcodeMLChildElements}

\begin{XcodeMLAttributes}
\XcodeMLAttrDef{common attributes}{-}
{Refer to "\ref{sec:Commonattributesofdefinition} Common attributes of the definition / declaration / statement element".}{-}
\end{XcodeMLAttributes}


\subsection{ {\tt FpragmaStatement} element}

The {\tt FpragmaStatement} element represents a directive of OpenMP 3.0 begining with ’{\tt !\$}’ or ’{\tt !\$OMP}’.
It has the content as text data.

%\subsubsection*{Contents model}
%{\tt
%(#PCDATA)
%}
\XcodeMLContentsModel{ (\#PCDATA) }

\begin{XcodeMLChildElements}
\XcodeMLElementDef{-}
{-}{-}
\end{XcodeMLChildElements}

\begin{XcodeMLAttributes}
\XcodeMLAttrDef{common attributes}{-}
{Refer to "\ref{sec:Commonattributesofdefinition} Common attributes of the definition / declaration / statement element".}{-}
\end{XcodeMLAttributes}


\subsection{ {\tt FcontainsStatement} element}

The {\tt FcontainsStatement} element represents a {\tt contains} statement.

%\subsubsection*{Contents model}
%{\tt
%(FfunctionDefinition+)
%}
\XcodeMLContentsModel{ (FfunctionDefinition+) }

\begin{XcodeMLChildElements}
\XcodeMLElementDef{FfunctionDefinition}
{Specifies the functions and subroutines included in the element.}{R}
\end{XcodeMLChildElements}

\begin{XcodeMLAttributes}
\XcodeMLAttrDef{common attributes}{-}
{Refer to "\ref{sec:Commonattributesofdefinition} Common attributes of the definition / declaration / statement element".}{-}
\end{XcodeMLAttributes}


\subsection{ {\tt condition} element}

The {\tt condition} element represents a conditional expression.

%\subsubsection*{Contents model}
%{\tt
%(exprModel)
%}
\XcodeMLContentsModel{ (exprModel) }

\begin{XcodeMLChildElements}
\XcodeMLElementDef{exprModel}
{Specifies the expression. Refer to "\ref{sec:exprModel} exprModel".}{R}
\end{XcodeMLChildElements}

\begin{XcodeMLAttributes}
\XcodeMLAttrDef{-}{-}
{-}{-}
\end{XcodeMLAttributes}


\subsection{ {\tt then} element}

The {\tt then} element represents a block executed if the condition falls true.

%\subsubsection*{Contents model}
%{\tt
%(body)
%}
\XcodeMLContentsModel{ (body) }

\begin{XcodeMLChildElements}
\XcodeMLElementDef{body}
{Specifies the statements included in the {\tt then} statement.}{R}
\end{XcodeMLChildElements}

\begin{XcodeMLAttributes}
\XcodeMLAttrDef{-}{-}
{-}{-}
\end{XcodeMLAttributes}


\subsection{ {\tt else} element}

The {\tt else} element represents a block executed if the condition falls false.

%\subsubsection*{Contents model}
%{\tt
%(body)
%}
\XcodeMLContentsModel{ (body) }

\begin{XcodeMLChildElements}
\XcodeMLElementDef{name}
{Specifies the statements included int the {\tt else} statement.}{R}
\end{XcodeMLChildElements}

\begin{XcodeMLAttributes}
\XcodeMLAttrDef{-}{-}
{-}{-}
\end{XcodeMLAttributes}


\subsection{ {\tt alloc} element}

The {\tt alloc} element represents an object in the allocation list of the {\tt allocate}
and {\tt deallocate} statement and in the pointer list of the {\tt nullify} statement.

%\subsubsection*{Contents model}
%{\tt
%((FmemberRef | Var), (arrayIndex | indexRange)?, coShape?*)
%}
\XcodeMLContentsModel{ ((FmemberRef | Var), (arrayIndex | indexRange)?, coShape?*) }

\begin{XcodeMLChildElements}
\XcodeMLElementDef{FmemberRef}
{Specifies the reference to the component of the structure.}{R}
\XcodeMLElementDef{Var}
{Specifies the variable.}{R}
\XcodeMLElementDef{arrayIndex}
{Specified if the type element is the array type and the size is expressed by the number of
 the elements. Can be repeated the number of the dimension times together with the {\tt indexRange} element.
 Neglected if the element is a child element of the {\tt FnullifyStatement}.}{O}
\XcodeMLElementDef{indexRange}
{Specified if the type element is the array type and the size is expressed by the upper and lower bounds.
 Can be repeated the number of the dimension times together with the {\tt arrayIndex} element.
 Neglected if the element is a child element of the {\tt FnullifyStatement}.}{O}
\XcodeMLElementDef{coShape}
{Specified if the element represents the allocation of a coarray.}{O}
\end{XcodeMLChildElements}

\begin{XcodeMLAttributes}
\XcodeMLAttrDef{-}{-}
{-}{-}
\end{XcodeMLAttributes}


\subsection{ {\tt allocOpt} element}

The {\tt allocOpt} element represents an {\tt alloc} option.

%\subsubsection*{Contents model}
%{\tt
%(exprModel+)
%}
\XcodeMLContentsModel{ (exprModel+) }

\begin{XcodeMLChildElements}
\XcodeMLElementDef{exprModel}
{Specifies the expression for the value of the element.}{R}
\end{XcodeMLChildElements}

\begin{XcodeMLAttributes}
\XcodeMLAttrDef{kind}{text}
{Specifies "{\tt errmsg}", "{\tt mold}", "{\tt source}" or "{\tt stat}" as the kind of the {\tt alloc} option.}{R}
\end{XcodeMLAttributes}


\subsection{ {\tt forallStatement} element}

The {\tt forAllStatement} element represents a {\tt forall} construct.

%\subsubsection*{Contents model}
%{\tt
%((var, indexRange)+, condition?, body)
%}
\XcodeMLContentsModel{ ((var, indexRange)+, condition?, body) }

\begin{XcodeMLChildElements}
\XcodeMLElementDef{var}
{Specifies the index variable.}{O}
\XcodeMLElementDef{indexRange}
{Specifies the value range of the index variable.}{O}
\XcodeMLElementDef{condition}
{Specifies the masking condition.}{O}
\XcodeMLElementDef{body}
{Specifies the content statements of the forall construct.}{R}
\end{XcodeMLChildElements}

\begin{XcodeMLAttributes}
\XcodeMLAttrDef{common attributes}{-}
{Refer to "\ref{sec:Commonattributesofdefinition} Common attributes of the definition / declaration / statement element".}{-}
\XcodeMLAttrDef{type}{text}
{Specifies the type identifier if the type is specified.}{O}
\XcodeMLAttrDef{construct\_name}{text}
{Specifies the construct name.}{O}
\end{XcodeMLAttributes}

\subsubsection*{Example}

The {\tt forallStatement} element for the {\tt forall} construct below is as follows:
\vspace{2mm}

\begin{Fexample2008}
forall (i=1:n,j=1:m,ix(i,j)>0)
 ix(i,j)=0
end forall
\end{Fexample2008}
\vspace{1mm}

\begin{XcodeMLFExample}
<forallStatement>
  <Var>i</Var>
  <indexRange>
    <lowerBound>
      <FintConstant>1</FintConstant>
    </lowerBound>
    <upperBound>
      <Var>n</Var>
    </upperBound>
    <step>
      <FintConstant>1</FintConstant>
    </step>
  </indexRange>
    <Var>j </Var>
  <indexRange>
    <lowerBound>
      <FintConstant>1</FintConstant>
    </lowerBound>
    <upperBound>
      <Var>m</Var>
    </upperBound>
    <step>
      <FintConstant>1</FintConstant>
    </step>
  </indexRange>
  <condition>
     <logGTExpr>
       <FarrayRef>
         <varRef>
           <Var>ix</Var>
         </varRef>
         <arrayIndex>
           <Var>i</Var>
         </arrayIndex>
         <arrayIndex>
           <Var>j</Var>
         </arrayIndex>
       </FarrayRef>
       <FintConstant>0</FintConstant>
     </logGTExpr>
   </condition>
  <body>
    <FassignStatement">
      <FarrayRef>
        <varRef>
          <Var>ix</Var>
        </varRef>
        <arrayIndex>
          <Var>i</Var>
        </arrayIndex>
        <arrayIndex>
          <Var>j</Var>
        </arrayIndex>
      </FarrayRef>
      <FintConstant>0</FintConstant>
    </FassignStatement>
  </body>
</FdoStatement>
\end{XcodeMLFExample}


\subsection{ {\tt FdoConcurrentStatement} element}

The {\tt FdoConcurrentStatement} element represents a {\tt do concurrent} statement.

%\subsubsection*{Contents model}
%{\tt
%((var, indexRange)+, condition?, body?)
%}
\XcodeMLContentsModel{ ((var, indexRange)+, condition?, body?) }

\begin{XcodeMLChildElements}
\XcodeMLElementDef{var}
{Specifies the {\tt do} variable.}{R}
\XcodeMLElementDef{indexRange}
{Specifies the value range of the {\tt do} variable.}{R}
\XcodeMLElementDef{condition}
{Specifies the masking condition.}{O}
\XcodeMLElementDef{body}
{Specifies the content statements.}{O}
\end{XcodeMLChildElements}

\begin{XcodeMLAttributes}
\XcodeMLAttrDef{common attributes}{-}
{Refer to "\ref{sec:Commonattributesofdefinition} Common attributes of the definition / declaration / statement element".}{-}
\XcodeMLAttrDef{type}{text}
{Specifies the type identifier if the type is specified.}{O}
\XcodeMLAttrDef{construct\_name}{text}
{Specifies the construct name.}{O}
\end{XcodeMLAttributes}


\subsection{ {\tt selectTypeStatement} element}

The {\tt selectTypeStatement} element represents a {\tt select type} construct.

%\subsubsection*{Contents model}
%{\tt
%(id, typeGuard*)
%}
\XcodeMLContentsModel{ (id, typeGuard*) }

\begin{XcodeMLChildElements}
\XcodeMLElementDef{id}
{Specifies the variable or the expression (and its associate name) to test the type.}{R}
\XcodeMLElementDef{typeGuard}
{Specifies the {\tt type guard} statement in the construct.}{R}
\end{XcodeMLChildElements}

\begin{XcodeMLAttributes}
\XcodeMLAttrDef{common attributes}{-}
{Refer to "\ref{sec:Commonattributesofdefinition} Common attributes of the definition / declaration / statement element".}{-}
\XcodeMLAttrDef{construct\_name}{text}
{Specifies the construct name.}{O}
\end{XcodeMLAttributes}

\subsubsection*{Example}

The {\tt selectTypeStatement} element for the {\tt select type} construct below is as follows:
\vspace{2mm}

\begin{Fexample2008}
class(t) x
...
select type(p=>x)
type is(t1)
  ...
type is(t2)
  ...
class is(t3)
  ...
type default
  ...
end select
\end{Fexample2008}
\vspace{1mm}

\begin{XcodeMLFExample}
<selectTypeStatement>
  <id>
    <name>p</name>
    <value>
       <Var>x</Var>
    </value>
  </id>
  <typeGuard kind="TYPE_IS" type="TYPE_ID_OF_T1">
    <body>
      ...
    </body>
  </typeGuard>
  <typeGuard kind="TYPE_IS" type="TYPE_ID_OF_T2">
    <body>
      ...
    </body>
  </typeGuard>
  <typeGuard kind="CLASS_IS" type="TYPE_ID_OF_T3">
    <body>
      ...
    </body>
  </typeGuard>
  <typeGuard kind="TYPE_DEFAULT">
    <body>
      ...
    </body>
  </typeGuard>
</selectTypeStatement>
\end{XcodeMLFExample}


\subsection{ {\tt typeGuard} element}

The {\tt typeGuard} element represents a {\tt type guard} statement in the {\tt select type} construct.

%\subsubsection*{Contents model}
%{\tt
%(body)
%}
\XcodeMLContentsModel{ (body) }

\begin{XcodeMLChildElements}
\XcodeMLElementDef{body}
{Specifies the content statements.}{R}
\end{XcodeMLChildElements}

\begin{XcodeMLAttributes}
\XcodeMLAttrDef{common attributes}{-}
{Refer to "\ref{sec:Commonattributesofdefinition} Common attributes of the definition / declaration / statement element".}{-}
\XcodeMLAttrDef{kind}{text}
{Specifies "{\tt TYPE\_IS}", "{\tt CLASS\_IS}" or "{\tt TYPE\_DEFAULT}" as the kind of the type guard.}{R}
\XcodeMLAttrDef{type}{text}
{Specifies the type identifier to test.}{O}
\end{XcodeMLAttributes}


\subsection{ {\tt syncAllStatement} element}

The {\tt syncAllStatement} element represents a {\tt sync all} statement.

%\subsubsection*{Contents model}
%{\tt
%(syncStat*)
%}
\XcodeMLContentsModel{ (syncStat*) }

\begin{XcodeMLChildElements}
\XcodeMLElementDef{syncStat}
{Specifies the stat specifiers.}{O}
\end{XcodeMLChildElements}

\begin{XcodeMLAttributes}
\XcodeMLAttrDef{common attributes}{-}
{Refer to "\ref{sec:Commonattributesofdefinition} Common attributes of the definition / declaration / statement element".}{-}
\end{XcodeMLAttributes}


\subsection{ {\tt syncImagesStatement} element}

The {\tt syncImagesStatement} element represents a {\tt sync image} statement.

%\subsubsection*{Contents model}
%{\tt
%(expModer?, syncStat*)
%}
\XcodeMLContentsModel{ (expModer?, syncStat*) }

\begin{XcodeMLChildElements}
\XcodeMLElementDef{exprModel}
{Specifies the expression representing the image. Omitted if '{\tt *}' is specified.}{O}
\XcodeMLElementDef{syncStat}
{Specifies the stat specifiers.}{O}
\end{XcodeMLChildElements}

\begin{XcodeMLAttributes}
\XcodeMLAttrDef{common attributes}{-}
{Refer to "\ref{sec:Commonattributesofdefinition} Common attributes of the definition / declaration / statement element".}{-}
\end{XcodeMLAttributes}


\subsection{ {\tt syncMemoryStatement} element}

The {\tt syncMemoryStatement} element represents a {\tt sync memory} statement.

%\subsubsection*{Contents model}
%{\tt
%(syncStat*)
%}
\XcodeMLContentsModel{ (syncStat*) }

\begin{XcodeMLChildElements}
\XcodeMLElementDef{syncStat}
{Specifies the stat specifiers.}{O}
\end{XcodeMLChildElements}

\begin{XcodeMLAttributes}
\XcodeMLAttrDef{common attributes}{-}
{Refer to "\ref{sec:Commonattributesofdefinition} Common attributes of the definition / declaration / statement element".}{-}
\end{XcodeMLAttributes}


\subsection{ {\tt lock/unlockStatement} element}

The {\tt lock/unlockStatement} element represents a {\tt lock/unlock} statement.

%\subsubsection*{Contents model}
%{\tt
%(Var?, syncStat*)
%}
\XcodeMLContentsModel{ (Var?, syncStat*) }

\begin{XcodeMLChildElements}
\XcodeMLElementDef{Var}
{Specifies the lock variable.}{O}
\XcodeMLElementDef{syncStat}
{Specifies the stat specifiers.}{O}
\end{XcodeMLChildElements}

\begin{XcodeMLAttributes}
\XcodeMLAttrDef{common attributes}{-}
{Refer to "\ref{sec:Commonattributesofdefinition} Common attributes of the definition / declaration / statement element".}{-}
\end{XcodeMLAttributes}


\subsection{ {\tt syncStat} element}

The {\tt syncStat} element represents a {\tt stat} specifier.

%\subsubsection*{Contents model}
%{\tt
%(Var)
%}
\XcodeMLContentsModel{ (Var) }

\begin{XcodeMLChildElements}
\XcodeMLElementDef{Var}
{Specifies the variable.}{O}
\end{XcodeMLChildElements}

\begin{XcodeMLAttributes}
\XcodeMLAttrDef{kind}{text}
{Specifies "{\tt STAT}", "{\tt ERRMSG}" or "{\tt ACQUIRED\_LOCK}" as the kind of the stat speficier.}{R}
\end{XcodeMLAttributes}


\subsection{ {\tt criticalStatement} element}

The {\tt criticalStatement} element represents a {\tt critical} construct.

%\subsubsection*{Contents model}
%{\tt
%(body)
%}
\XcodeMLContentsModel{ (body) }

\begin{XcodeMLChildElements}
\XcodeMLElementDef{body}
{Specifies the content statements.}{R}
\end{XcodeMLChildElements}

\begin{XcodeMLAttributes}
\XcodeMLAttrDef{common attributes}{-}
{Refer to "\ref{sec:Commonattributesofdefinition} Common attributes of the definition / declaration / statement element".}{-}
\XcodeMLAttrDef{construct\_name}{text}
{Specifies the construct name.}{O}
\end{XcodeMLAttributes}


\subsection{ {\tt associateStatement} element}

The {\tt associateStatement} element represents a {\tt associate} construct.

%\subsubsection*{Contents model}
%{\tt
%(symbols, body)
%}
\XcodeMLContentsModel{ (symbols, body) }

\begin{XcodeMLChildElements}
\XcodeMLElementDef{symbols}
{Specifies the bindings within the construct.}{R}
\XcodeMLElementDef{body}
{Specifies the content statements.}{R}
\end{XcodeMLChildElements}

\begin{XcodeMLAttributes}
\XcodeMLAttrDef{common attributes}{-}
{Refer to "\ref{sec:Commonattributesofdefinition} Common attributes of the definition / declaration / statement element".}{-}
\XcodeMLAttrDef{construct\_name}{text}
{Specifies the construct name.}{O}
\end{XcodeMLAttributes}

\subsubsection*{Example}

The {\tt associateStatement} element for the {\tt associate} statement below is as follows:
\vspace{2mm}

\begin{Fexample2008}
associate(x=>... , y=>...)
 ... = x
 ... = y
end associate
\end{Fexample2008}
\vspace{1mm}

\begin{XcodeMLFExample}
<associateStatement>
  <symbols>
    <id>
      <name>x</name>
      <value>
        ...
      </value>
    </id>
    <id>
      <name>y</name>
      <value>
        ...
      </value>
    </id>
  </symbols>
  <body>
     ...
  </body>
</associateStatement>
\end{XcodeMLFExample}


\subsection{ {\tt blockStatement} element}

The {\tt blockStatement} element represents a {\tt block} construct.

%\subsubsection*{Contents model}
%{\tt
%(symbols?, declarations?, body)
%}
\XcodeMLContentsModel{ (symbols?, declarations?, body) }

\begin{XcodeMLChildElements}
\XcodeMLElementDef{symbols}
{Specifies the symbols included in the element.}{O}
\XcodeMLElementDef{declarations}
{Specifies the declarations included in the block construct.}{O}
\XcodeMLElementDef{body}
{Specifies the content statements.}{R}
\end{XcodeMLChildElements}

\begin{XcodeMLAttributes}
\XcodeMLAttrDef{common attributes}{-}
{Refer to "\ref{sec:Commonattributesofdefinition} Common attributes of the definition / declaration / statement element".}{-}
\XcodeMLAttrDef{construct\_name}{text}
{Specifies the construct name.}{O}
\end{XcodeMLAttributes}

\subsubsection*{Example}

The {\tt blockStatement} element for the {\tt block} construct below is as follows:
\vspace{2mm}

\begin{Fexample2008}
block
 integer i
 real x
  ...
end block
\end{Fexample2008}
\vspace{1mm}

\begin{XcodeMLFExample}
<blockStatement>
  <symbols>
     <id type="Fint" sclass="flocal">
       <name>i</name>
     </id>
     <id type="Freal" sclass="flocal">
       <name>a</name>
     </id>
   </symbols>
   <declarations>
     <varDecl>
       <name type="Fint">i</name>
     </varDecl>
     <varDecl>
       <name type="Freal">a</name>
     </varDecl>
  </declarations>
  <body>
     ...
  </body>
</blockStatement>
\end{XcodeMLFExample}


