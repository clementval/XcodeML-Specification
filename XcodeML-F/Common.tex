\section{Common element}

The definitions commonly used in an arbitrary element are shown below.

\subsection{The data type name identifier}
\label{sec:Thedatatypenameidentifier}

In the program, the data type is identified by the data name.
The name is one of the basic data type names or the derived data type names below.
\newline

\begin{XcodeMLDataTypeNames}
\XcodeMLDataTypeName{Fint}{integer type}
\XcodeMLDataTypeName{Freal}{real type}
\XcodeMLDataTypeName{Fcomplex}{complex type}
\XcodeMLDataTypeName{Flogical}{logical type}
\XcodeMLDataTypeName{Fcharacter}{character type}
\XcodeMLDataTypeName{Fvoid}{void type; represents the type of the return value of the subroutine.}
\XcodeMLDataTypeName{others}
{derived type; expressed by an arbitrary string consisting of alphabets and numbers
 different from the names of the basic data types. It must be unique in the program.}
\end{XcodeMLDataTypeNames}


\subsection{Common attributes of the definition / declaration / statement element}
\label{sec:Commonattributesofdefinition}

Every definition, declaration and statement elements may have the following attributes.

\begin{XcodeMLAttributes}
\XcodeMLAttrDef{lineno}{text}
{Specifies the line number in the Fortran program.}{R}
\XcodeMLAttrDef{file}{text}
{Specifies the source code name of the Fortran program.}{R}
\end{XcodeMLAttributes}


\subsection{ {\tt lValueModel} model}
\label{sec:lValueModel}

The {\tt lValueModel model} is commonly used by the elements which refer to the variable as the lvalue.

%\subsubsection*{Contents model}
%{\tt
%(Var | FarrayRef | FcharacterRef | FmemberRef | FcoArrayRef)
%}
\XcodeMLContentsModel{ (Var | FarrayRef | FcharacterRef | FmemberRef | FcoArrayRef) }

\subsubsection*{Element}

Refer to the specification of each element.


\subsection{ {\tt exprModel} model}
\label{sec:exprModel}

The {\tt exprModel} model is commonly used by the elements which refer to the expression.

%\subsubsection*{Contents model}
%{\tt
%(FintConstant | FrealConstant | FcomplexConstant | FcharacterConstant | FlogicalConstant | FarrayConstructor | FstructConstructor | Var | FarrayRef | FcharacterRef | FmemberRef | FcoArrayRef | varRef | functionCall | plusExpr | minusExpr | mulExpr | divExpr | FpowerExpr | FconcatExpr | logEQExpr | logNEQExpr | logGEExpr | logGTExpr | logLEExpr | logLTExpr | logAndExpr | logOrExpr | logEQVExpr | logNEQVExpr | unaryMinusExpr | logNotExpr | userBinaryExpr | userUnaryExpr | FdoLoop)
%}
\XcodeMLContentsModel{ (FintConstant | FrealConstant | FcomplexConstant | FcharacterConstant | FlogicalConstant | FarrayConstructor | FstructConstructor | Var | FarrayRef | FcharacterRef | FmemberRef | FcoArrayRef | varRef | functionCall | plusExpr | minusExpr | mulExpr | divExpr | FpowerExpr | FconcatExpr | logEQExpr | logNEQExpr | logGEExpr | logGTExpr | logLEExpr | logLTExpr | logAndExpr | logOrExpr | logEQVExpr | logNEQVExpr | unaryMinusExpr | logNotExpr | userBinaryExpr | userUnaryExpr | FdoLoop) }

\subsubsection*{Element}

Refer to the specification of each element.


\subsection{ {\tt statementModel} model}

The {\tt statementModel} model is commonly used by the elements which refer to the statement.

%\subsubsection*{Contents model}
%{\tt
%(FassignStatement | exprStatement | FpointerAssignStatement | FifStatement | FdoStatement | FdoWhileStatement | continueStatement | FcycleStatement | FexitStatement | FreturnStatement | gotoStatement | statementLabel | FselectCaseStatement | FcaseLabel | FwhereStatement | FstopStatement | FreadStatement | FwriteStatement | FprintStatement | FrewindStatement | FendFileStatement | FbackspaceStatement | FopenStatement | FcloseStatement | FinquireStatement | FformatDecl | FdataDecl | FentryDecl | FallocateStatement | FdeallocateStatement | FnullifyStatement | FpragmaStatement | FcontainsStatement | forAllStatement | FdoConcurrentStatement | selectTypeStatement | associateStatement | blockStatement)
%}
\XcodeMLContentsModel{ (FassignStatement | exprStatement | FpointerAssignStatement | FifStatement | FdoStatement | FdoWhileStatement | continueStatement | FcycleStatement | FexitStatement | FreturnStatement | gotoStatement | statementLabel | FselectCaseStatement | FcaseLabel | FwhereStatement | FstopStatement | FreadStatement | FwriteStatement | FprintStatement | FrewindStatement | FendFileStatement | FbackspaceStatement | FopenStatement | FcloseStatement | FinquireStatement | FformatDecl | FdataDecl | FentryDecl | FallocateStatement | FdeallocateStatement | FnullifyStatement | FpragmaStatement | FcontainsStatement) }

\subsubsection*{Element}

Refer to the specification of each element.


