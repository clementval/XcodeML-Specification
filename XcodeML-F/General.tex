\section{General element}


\subsection{ {\tt kind} element}

The {\tt kind} element represents a {\tt kind} parameter of the type.

%\subsubsection*{Contents model}
%{\tt
%(\#PCDATA)
%}
\XcodeMLContentsModel{ (\#PCDATA) }

\begin{XcodeMLChildElements}
\XcodeMLElementDef{-}
{-}{-}
\end{XcodeMLChildElements}

\begin{XcodeMLAttributes}
\XcodeMLAttrDef{-}{-}
{-}{-}
\end{XcodeMLAttributes}


\subsection{ {\tt id} element}

The {\tt id} element represents an identifier of the name of the variable, array,
structure component, dummy argument of the function or subroutine, etc.

%\subsubsection*{Contents model}
%{\tt
%(name)
%}
\XcodeMLContentsModel{ (name) }

\begin{XcodeMLChildElements}
\XcodeMLElementDef{name}
{Specifies the name of the identifier(the name of the variable if the identifier corresponds to a variable etc.).}{R}
\end{XcodeMLChildElements}

\begin{XcodeMLAttributes}
\XcodeMLAttrDef{sclass}{text}
{Specifies the storage class, one of '{\tt auto}', '{\tt param}', '{\tt extern}', '{\tt extern\_def}', '{\tt label}', '{\tt tagname}'.}{R}
\XcodeMLAttrDef{type}{text}
{Specifies the type name identifier in XcodeML/Fortran.
 Refer to "\ref{sec:Thedatatypenameidentifier} The data type name identifier".}{R}
\end{XcodeMLAttributes}

\subsubsection*{Example}

The symbol table entry of the variable "{\tt xyz}" in "{\tt integer xyz}" is as follows:
\vspace{2mm}

\begin{XcodeMLFExample}
<id sclass="extern_def" type="Fint">
  <name>xyz</name>
</id>
\end{XcodeMLFExample}

The symbol table entry of the function "{\tt foo}" in "{\tt function foo}" is as follows:
Here, "{\tt F06f168}" is the {\tt type\_id} of the data type of "{\tt foo}".
\vspace{2mm}

\begin{XcodeMLFExample}
<id sclass="extern_def" type="F06f168">
  <name>foo</name>
</id>
\end{XcodeMLFExample}


\subsection{ {\tt name} element}

The {\tt name} element specifies the name of a variable or a type name etc.

%\subsubsection*{Contents model}
%{\tt
%(\#PCDATA)
%}
\XcodeMLContentsModel{ (\#PCDATA) }

\begin{XcodeMLChildElements}
\XcodeMLElementDef{-}
{-}{-}
\end{XcodeMLChildElements}

\begin{XcodeMLAttributes}
\XcodeMLAttrDef{type}{text}
{Specifies the type name identifier in XcodeML/Fortran.}{R}
\end{XcodeMLAttributes}


\subsection{ {\tt value} element}

The {\tt value} element represents an arbitrary value expressed by an expression.

%\subsubsection*{Contents model}
%{\tt
%(exprModel)
%}
\XcodeMLContentsModel{ (exprModel) }

\begin{XcodeMLChildElements}
\XcodeMLElementDef{exprModel}
{Specifies the expression of the value.
 Refer to "\ref{sec:exprModel} exprModel".}{R}
\end{XcodeMLChildElements}

\begin{XcodeMLAttributes}
\XcodeMLAttrDef{repeat\_counttyp}{text}
{Specifies the repeat count of the element.
 Valid only in the initialization expression of the {\tt data} statement.}{O}
\end{XcodeMLAttributes}


\subsection{ {\tt kind} params}

The {\tt param} element represents a dummy argument list of the function or subroutine.
The element is also used to represent a statement label list in the {\tt go to} statement.

%\subsubsection*{Contents model}
%{\tt
%(name*)
%}
\XcodeMLContentsModel{ (name*) }

\begin{XcodeMLChildElements}
\XcodeMLElementDef{name}
{Specifies the names of the dummy arguments or the statement labels.
 In the case of the statement labels, the type attribute of the name element must be '{\tt Fint}'.}{O}
\end{XcodeMLChildElements}

\begin{XcodeMLAttributes}
\XcodeMLAttrDef{-}{-}
{-}{-}
\end{XcodeMLAttributes}


\subsection{ {\tt len} element}

The {\tt len} element represents an arbitrary length expressed by an expression.

%\subsubsection*{Contents model}
%{\tt
%(exprModel)
%}
\XcodeMLContentsModel{ (exprModel) }

\begin{XcodeMLChildElements}
\XcodeMLElementDef{exprModel}
{Specifies the expression of the length. Refer to "\ref{sec:exprModel} exprModel".}{R}
\end{XcodeMLChildElements}

\begin{XcodeMLAttributes}
\XcodeMLAttrDef{is\_assumed\_shape}{bool}
{{\tt true} if no length is specified("{\tt (:)}").}{O}
\XcodeMLAttrDef{is\_assume\_size}{bool}
{{\tt true} if the length is the assumed-size.}{O}
\end{XcodeMLAttributes}


\subsection{ {\tt body} element}

The {\tt body} element represents a block of statements.

%\subsubsection*{Contents model}
%{\tt
%(statementModel*)
%}
\XcodeMLContentsModel{ (statementModel*) }

\begin{XcodeMLChildElements}
\XcodeMLElementDef{statementModel}
{Specifies arbitrary statements.}{O}
\end{XcodeMLChildElements}

\begin{XcodeMLAttributes}
\XcodeMLAttrDef{-}{-}
{-}{-}
\end{XcodeMLAttributes}


\subsection{ {\tt rename} element}

The {\tt rename} element represents a local name and a name accessed in the {\tt use} statement.

%\subsubsection*{Contents model}
%{\tt
%empty
%}
\XcodeMLContentsModel{ empty }

\begin{XcodeMLChildElements}
\XcodeMLElementDef{-}
{-}{-}
\end{XcodeMLChildElements}

\begin{XcodeMLAttributes}
\XcodeMLAttrDef{use\_name}{text}
{Specifies the name accessed.}{R}
\XcodeMLAttrDef{local\_name}{text}
{Specifies the local name.}{R}
\XcodeMLAttrDef{is\_operator}{bool}
{{\tt true} if the element represents the rename of the defined operator.}{O}
\end{XcodeMLAttributes}


\subsection{ {\tt renamable} element}

The {\tt renamable} element represents a local name and a name accessed, which are omissible, in the {\tt use} statement with the only option.

%\subsubsection*{Contents model}
%{\tt
%empy
%}
\XcodeMLContentsModel{ empy }

\begin{XcodeMLChildElements}
\XcodeMLElementDef{-}
{-}{-}
\end{XcodeMLChildElements}

\begin{XcodeMLAttributes}
\XcodeMLAttrDef{use\_name}{text}
{Specifies the name accessed.}{R}
\XcodeMLAttrDef{local\_name}{text}
{Specifies the local name.}{O}
\XcodeMLAttrDef{is\_operator}{bool}
{{\tt true} if the element represents the rename of the defined operator.}{O}
\end{XcodeMLAttributes}


\subsection{ {\tt arrayIndex} element}

The {\tt arrayIndex} element represents an arbitrary index value(or size) expressed by an expression.

%\subsubsection*{Contents model}
%{\tt
%(exprModel)
%}
\XcodeMLContentsModel{ (exprModel) }

\begin{XcodeMLChildElements}
\XcodeMLElementDef{exprModel}
{Specifies the expression of the value. Refer to "\ref{sec:exprModel} exprModel".}{R}
\end{XcodeMLChildElements}

\begin{XcodeMLAttributes}
\XcodeMLAttrDef{-}{-}
{-}{-}
\end{XcodeMLAttributes}


\subsection{ {\tt indexRange} element}

The {\tt indexRange} element represents an arbitrary range of the subscript(or size) expressed by an upper and lower bound and a step.
The meaning of the element depends on the kind of the parent element.

%\subsubsection*{Contents model}
%{\tt
%(lowerBound?, upperBound?, step?)
%}
\XcodeMLContentsModel{ (lowerBound?, upperBound?, step?) }

\begin{XcodeMLChildElements}
\XcodeMLElementDef{lowerBound}
{Specifies the lower bound. The kind of the parent element determines the default value.}{O}
\XcodeMLElementDef{upperBound}
{Specifies the upper bound. The kind of the parent element determines the default value.}{O}
\XcodeMLElementDef{step}
{Specifies the step. May be neglected depending on the kind of the parent element.}{O}
\end{XcodeMLChildElements}

\begin{XcodeMLAttributes}
\XcodeMLAttrDef{is\_assumed\_shape}{bool}
{{\tt true} in the case of the assumed-shape array.}{O}
\XcodeMLAttrDef{is\_assumed\_size}{bool}
{{\tt true} in the case of the assumed-size array.}{O}
\end{XcodeMLAttributes}


\subsection{ {\tt lowerBound} element}

The {\tt lowerBound} element represents a lower bound of a range.

%\subsubsection*{Contents model}
%{\tt
%(exprModel)
%}
\XcodeMLContentsModel{ (exprModel) }

\begin{XcodeMLChildElements}
\XcodeMLElementDef{exprModel}
{Specifies the expression of the lower bound. Refer to "\ref{sec:exprModel} exprModel".}{R}
\end{XcodeMLChildElements}

\begin{XcodeMLAttributes}
\XcodeMLAttrDef{-}{-}
{-}{-}
\end{XcodeMLAttributes}


\subsection{ {\tt upperBound} element}

The {\tt upperBound} element represents a upper bound of the range.

%\subsubsection*{Contents model}
%{\tt
%(exprModel)
%}
\XcodeMLContentsModel{ (exprModel) }

\begin{XcodeMLChildElements}
\XcodeMLElementDef{exprModel}
{Specifies the expression of the upper bound. Refer to "\ref{sec:exprModel} exprModel".}{R}
\end{XcodeMLChildElements}

\begin{XcodeMLAttributes}
\XcodeMLAttrDef{-}{-}
{-}{-}
\end{XcodeMLAttributes}


\subsection{ {\tt step} element}

The {\tt step} element represents a stride of a range.

%\subsubsection*{Contents model}
%{\tt
%(exprModel)
%}
\XcodeMLContentsModel{ (exprModel) }

\begin{XcodeMLChildElements}
\XcodeMLElementDef{exprModel}
{Specifies the expression of the stride. Refer to "\ref{sec:exprModel} exprModel".}{R}
\end{XcodeMLChildElements}

\begin{XcodeMLAttributes}
\XcodeMLAttrDef{-}{-}
{-}{-}
\end{XcodeMLAttributes}


\subsection{ {\tt FdoLoop} element}

The {\tt FdoLoop} element represents an implied-do.
The element defines all the implied-does in the {\tt data} statement, the I/O statement and the array constructor.

%\subsubsection*{Contents model}
%{\tt
%(Var, indexRange, value+)
%}
\XcodeMLContentsModel{ (Var, indexRange, value+) }

\begin{XcodeMLChildElements}
\XcodeMLElementDef{Var}
{Specifies the {\tt do} variable.}{R}
\XcodeMLElementDef{indexRange}
{Specifies the upper and lower bounds.}{R}
\XcodeMLElementDef{value}
{Specifies the expression element for the value of the implied-do.}{R}
\end{XcodeMLChildElements}

\begin{XcodeMLAttributes}
\XcodeMLAttrDef{-}{-}
{-}{-}
\end{XcodeMLAttributes}


\subsection{ {\tt namedValue} element}

The {\tt namedValue} element represents a value with a keyword.

%\subsubsection*{Contents model}
%{\tt
%empty
%}
\XcodeMLContentsModel{ empty }

\begin{XcodeMLChildElements}
\XcodeMLElementDef{-}
{-}{-}
\end{XcodeMLChildElements}

\begin{XcodeMLAttributes}
\XcodeMLAttrDef{name}{text}
{Specifies the keyword.}{R}
\XcodeMLAttrDef{value}{text}
{Specifies the value.}{R}
\end{XcodeMLAttributes}


\subsection{ {\tt namedValueList} element}

The {\tt namedValueList} element represents a value list with keywords.

%\subsubsection*{Contents model}
%{\tt
%(namedValue*)
%}
\XcodeMLContentsModel{ (namedValue*) }

\begin{XcodeMLChildElements}
\XcodeMLElementDef{namedValue}
{Specifies the value with the keyword.}{O}
\end{XcodeMLChildElements}

\begin{XcodeMLAttributes}
\XcodeMLAttrDef{-}{-}
{-}{-}
\end{XcodeMLAttributes}


\subsection{ {\tt valueList} element}

The {\tt valueList} element represents a value list expressed by an expression.

%\subsubsection*{Contents model}
%{\tt
%(value*)
%}
\XcodeMLContentsModel{ (value*) }

\begin{XcodeMLChildElements}
\XcodeMLElementDef{value}
{Specifies the expression for the value.}{O}
\end{XcodeMLChildElements}

\begin{XcodeMLAttributes}
\XcodeMLAttrDef{-}{-}
{-}{-}
\end{XcodeMLAttributes}


\subsection{ {\tt varList} element}

The {\tt varList} element represents a list.

%\subsubsection*{Contents model}
%{\tt
%(varRef | FdoLoop)*
%}
\XcodeMLContentsModel{ (varRef | FdoLoop)* }

\begin{XcodeMLChildElements}
\XcodeMLElementDef{varRef}
{Specifies the item expressed by the reference to the variable.}{O}
\XcodeMLElementDef{FdoLoop}
{Specifies the list of the items expressed by the implied-do.}{O}
\end{XcodeMLChildElements}

\begin{XcodeMLAttributes}
\XcodeMLAttrDef{name}{text}
{Specifies the {\tt namelist} group name if the parent is the {\tt FnamelistDecl} element.}{O}
\end{XcodeMLAttributes}


