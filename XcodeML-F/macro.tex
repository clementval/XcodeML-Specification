\def\openb{{\it [}}
\def\closeb{{\it ]}}
%\def\openp{{\tt (}}
%\def\closep{{\tt )}}

\def\DAG{$^\dagger$}
\def\DDAG{$^\ddagger$}

\def\XMP{XcalableMP}
\def\OMP{OpenMP}
\def\HPF{HPF}
\def\CAF{Co-array Fortran}
\def\MPI{MPI}
\def\Fort{Fortran}
\def\FortN{Fortran 90}
\def\FortE{Fortran 2008}
\def\C{C}
\def\XMPF{XcalableMP Fortran}
\def\XMPC{XcalableMP C}

\def\XcodeML{XcodeML}
\def\XcodeMLF{XcodeML/Fortran}

\def\Directive#1{{\tt #1}\index{#1@{\tt #1}}\index{Directive!#1@{\tt #1}}}

%\def\Syntax#1{\index{{\tt #1}}\index{Syntax!{\tt #1}}}
\def\Syntax#1{\index{Syntax!#1@{\tt #1}}}

\def\Term#1{{#1}\index{#1}}

%\def\Example#1{\index{#1}\index{Example!{\tt #1}}}
\def\Example#1{\index{Example!#1@{\tt #1}}}

\def\Intrinsic#1{\index{#1@{\tt #1}}\index{Intrinsic and Library Procedures!#1@{\tt #1}}}

\def\NULL{{\tt NULL}}

\def\PROG#1{{\tt #1}\index{\tt #1}\index{Command!#1}}
\def\FUNC#1{{\tt #1()}\index{\tt #1}\index{Example Function!#1}}
\def\LPROG#1{{\tt #1}\index{\tt #1}\index{Linux Command!#1}}
\def\LFUNC#1{{\tt #1()}\index{\tt #1}\index{Linux Function!#1}}
\def\MFUNC#1{{\tt #1()}\index{\tt #1}\index{Myrinet/MX Function!#1}}
\def\SFUNC#1{{\tt #1()}\index{\tt #1}\index{Sample Code!#1}}
\def\STRUCT#1{{\tt #1}\index{\tt #1}\index{Struct!#1}}
\def\VAR#1{{\tt #1}\index{\tt #1}\index{Variable!#1}}
\def\ARG#1{{\tt #1}\index{\tt #1}\index{Variable!#1}}
\def\MACRO#1{{\tt #1}\index{\tt #1}\index{Macro!#1}}
\def\ERRNO#1{{\tt #1}\index{\tt #1}\index{ERRNO!#1}}
\def\SIGNAL#1{{\tt #1}\index{\tt #1}\index{Signal!#1}}
\def\FILE#1{{\tt #1}\index{\tt #1}\index{File!#1}}
\def\ENV#1{{\tt #1}\index{\tt #1}\index{Environment Variable!#1}}
\def\INDEX#1{#1\index{#1}}
\def\OPTION#1{{\tt #1}\index{Command Option!#1}}
\def\TERM#1{\underline{#1}\index{#1}}
\def\CTRL#1{{\tt $^\wedge$#1}}

\def\phrule{\vspace{0.2cm}\hrule\vspace{0.05cm}\hrule}
\def\qhrule{\vspace{0.2cm}\hrule}

\def\dhrule{\hrule\vspace{0.05cm}\hrule}

\def\bsquare{\rule[-2pt]{5pt}{10pt}}

\newcommand\gio{{\tt global\_io }}
\newcommand\mio{{\tt master\_io }}

%\newcommand{\mytextcolor}[2]{\textcolor{#1}{#2}}
\newcommand{\mytextcolor}[2]{{#2}}
\newcommand{\mycolor}[1]{\color{#1}}

\newenvironment{point}[1]{\vspace*{0.3cm}\begin{itembox}{#1}}{\end{itembox}\vspace*{0.3cm}}

\newenvironment{note}[1]{\vspace*{0.3cm}\begin{itembox}{Note on #1}}{\end{itembox}\vspace*{0.3cm}}

\newenvironment{issue}[1]{\vspace*{0.3cm}\begin{itembox}{Issues on #1}}{\end{itembox}\vspace*{0.3cm}}

\newenvironment{errors}{\vspace*{0.3cm}\begin{tabular}{ll}\multicolumn{2}{l}{\bf Return Values}\\}{\end{tabular}\vspace*{0.3cm}}

\newenvironment{mytable}[3]{\begin{table}[ht]\caption{#1}\label{#2}\vspace*{-0.3cm}\begin{center}\begin{tabular}{#3}}{\end{tabular}\end{center}\end{table}}

\newenvironment{myfigure}{\begin{figure}[ht]\begin{center}}{\end{center}\end{figure}}

\DefineVerbatimEnvironment{Fexample}{Verbatim}{numbers=left,numbersep=3pt,stepnumber=5,frame=single,label=\Fort}

\DefineVerbatimEnvironment{FexampleR}{Verbatim}{numbers=right,numbersep=3pt,stepnumber=5,frame=single,label=\Fort}

\DefineVerbatimEnvironment{Fexample90}{Verbatim}{numbers=left,numbersep=3pt,stepnumber=5,frame=single,label=\FortN}

\DefineVerbatimEnvironment{Fexample2008}{Verbatim}{numbers=left,numbersep=3pt,stepnumber=5,frame=single,label=\FortE}

\DefineVerbatimEnvironment{Cexample}{Verbatim}{numbers=left,numbersep=3pt,stepnumber=5,frame=single,label=\C}

\DefineVerbatimEnvironment{CexampleR}{Verbatim}{numbers=right,numbersep=3pt,stepnumber=5,frame=single,label=\C}

\DefineVerbatimEnvironment{XFexample}{Verbatim}{numbers=left,numbersep=3pt,stepnumber=5,frame=single,label=\XMPF}

\DefineVerbatimEnvironment{XFexampleR}{Verbatim}{numbers=right,numbersep=3pt,stepnumber=5,frame=single,label=\XMPF}

\DefineVerbatimEnvironment{XCexample}{Verbatim}{numbers=left,numbersep=3pt,stepnumber=5,frame=single,label=\XMPC}

\DefineVerbatimEnvironment{XCexampleR}{Verbatim}{numbers=right,numbersep=3pt,stepnumber=5,frame=single,label=\XMPC}

\DefineVerbatimEnvironment{XcodeMLExample}{Verbatim}{numbers=left,numbersep=3pt,stepnumber=5,frame=single,label=\XcodeML}

\DefineVerbatimEnvironment{XcodeMLFExample}{Verbatim}{numbers=left,numbersep=3pt,stepnumber=5,frame=single,label=\XcodeMLF}

\DefineVerbatimEnvironment{CExample}{Verbatim}{numbers=left,numbersep=3pt,stepnumber=5,frame=single,label={\tt C code}}

%
% for XcodeML definition
%

\newcommand{\XcodeMLContentsModel}[1]{
\subsubsection*{Contents model}
{\tt #1}}

\newenvironment{XcodeMLChildElements}{
\subsubsection*{Child elements}
\begin{tabular}{|l|p{10cm}|c|}
\hline
 name & description & R/O \\ \hline\hline
}{\end{tabular}}

\newenvironment{XcodeMLElementDef}[3]{
{\tt #1} & #2 & #3 \\ \hline }

\newenvironment{XcodeMLAttributes}{
\subsubsection*{Attributes}
\begin{tabular}{|l|c|p{10cm}|c|}
\hline
name & type & description & R/O \\ \hline\hline
}{\end{tabular}}

\newenvironment{XcodeMLAttrDef}[4]{
{\tt #1} & #2 & #3 & #4\\ \hline }

\newenvironment{XcodeMLControlList}{
\subsubsection*{Control list}
The values for the name attribute and the child element of the {\tt namedValue} element are as follows:
\newline
\newline
\begin{tabular}{|l|p{10cm}|c|}
\hline
 name attribute & child element & R/O \\ \hline\hline
}{\end{tabular}}

\newenvironment{XcodeMLSpecifier}[3]{
{\tt #1} & #2 & #3 \\ \hline }

\newenvironment{XcodeMLElementList}{
\begin{tabular}{|l|p{10cm}|}
\hline
 element & format of the content \\ \hline\hline
}{\end{tabular}}

\newenvironment{XcodeMLElementFormat}[2]{
{\tt #1} & #2 \\ \hline }

\newenvironment{XcodeMLOperations}{
\begin{tabular}{|l|p{4cm}|l|}
\hline
 element & operator & operation \\ \hline\hline
}{\end{tabular}}

\newenvironment{XcodeMLOperation}[3]{
{\tt #1} & #2 & #3 \\ \hline }

\newenvironment{XcodeMLDataTypeNames}{
\begin{tabular}{|l|p{11cm}|}
\hline
 type name & description \\ \hline\hline
}{\end{tabular}}

\newenvironment{XcodeMLDataTypeName}[2]{
{\tt #1} & #2 \\ \hline }

